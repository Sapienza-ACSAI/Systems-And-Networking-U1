\documentclass[openright, twoside]{report}

\usepackage[utf8]{inputenc}

\usepackage[a4paper, right = 2.4cm, left=3cm, bottom = 2.5cm, top = 2.5cm, marginparwidth=2.2cm, marginparsep=2mm, heightrounded]{geometry}
\usepackage{amsmath}
\usepackage{amsfonts}
\usepackage{amsthm}
\usepackage{titlesec}
\usepackage{enumerate}
\usepackage{marginnote}
\usepackage{fancyvrb}

\reversemarginpar

\usepackage{listings}
\usepackage{quotchap}
\usepackage{epigraph}
\usepackage{xparse}
\usepackage[binary-units]{siunitx}
\usepackage{shellesc}
\usepackage{emptypage}
\usepackage{multicol}
\usepackage[para]{footmisc}
\usepackage{bookmark}
%%%
%
% Defining Fancy Header
%
%%%

\usepackage{fancyhdr, lastpage}
\usepackage{titlesec}
\pagestyle{fancy}

\fancyhead[LE,RO]{Dario Loi}
\fancyhead[RE,LO]{Operating Systems}

\fancyfoot[CE,CO]{\leftmark}
\fancyfoot[LE,RO]{\thepage \ of \pageref*{LastPage}}

\renewcommand{\headrulewidth}{0.5pt}

\fancypagestyle{plain}{
  \renewcommand{\headrulewidth}{0pt}
  \fancyhf{}
  \fancyfoot[CE,CO]{\leftmark}
  \fancyfoot[LE,RO]{\thepage \ of \pageref*{LastPage}}
}

%% baskerville

\usepackage[T1]{fontenc}
\usepackage{kpfonts,baskervald}
\usepackage{microtype}

%%%
%
% Defining Code Environment Typesetting
%
%%%

\lstset{language=C++,
                basicstyle=\ttfamily,
                keywordstyle=\color{blue}\ttfamily,
                stringstyle=\color{red}\ttfamily,
                commentstyle=\color{green}\ttfamily,
                morecomment=[l][\color{magenta}]{\#}
}


\DeclareTextFontCommand{\emph}{\bfseries}

\usepackage{hyperref}
  \hypersetup{
      colorlinks=true,
      linkcolor=blue,
      filecolor=magenta,      
      urlcolor=cyan,
      pdftitle={Systems and Networking Unit I},
      pdfpagemode=,
      }

\setcounter{secnumdepth}{3}
\setcounter{tocdepth}{3}
\usepackage{cleveref} 

\titleformat{\chapter}[display]
  {\normalfont\bfseries}{}{0pt}{\Large}
  
\theoremstyle{definition}
\newtheorem{definition}{Definition}[section]

\theoremstyle{example}
\newtheorem{example}{Example}[section]


%%%
%
% PGFPlots Config and TiKz macros
%
%%%

\usepackage{pgfplots}

\pgfplotsset{compat=1.18}


\newcommand{\vertLineFromPoint}[1]{
  \draw[dashed] 
  (#1) -- (#1|-{rel axis cs:0,0})
}
\newcommand{\horLineFromPoint}[1]{
  \draw[dashed] 
  (#1) -- (#1-|{rel axis cs:0,0})
}

\NewDocumentCommand{\plotmulti}{ O{-12} O{12} O{50} m m }{

  \begin{tikzpicture}
    \begin{axis}[xscale = 0.6, yscale=0.65]
      xmax = 13,
      xmin = -13,
      ymax = 13, 
      ymin = -13, 
      zmax = 50, 
      zmin = -10,
      grid = both]
      \addplot3[
        domain=#1:#2,
        samples = #3,
        surf]{#4};\label{#5}
    \end{axis}
  \end{tikzpicture}
}

\NewDocumentCommand{\plottwo}{ O{-12} O{12} O{50} m m m m}{

  \begin{tikzpicture}
    \begin{axis}[xscale = 0.6, yscale=0.65]
      xmax = 13,
      xmin = -13,
      ymax = 13, 
      ymin = -13, 
      grid = both]
      \addplot[
        domain=#1:#2,
        samples = #3,
        thick, red]{#4};\label{#5}
      \addplot[
      domain=#1:#2,
      samples = #3,
      thin, blue]{#6};\label{#7}
    \end{axis}
  \end{tikzpicture}

}

\NewDocumentCommand{\plotf}{ O{-12} O{12} O{50} m m}{

  \begin{tikzpicture}
    \begin{axis}[xscale = 0.6, yscale=0.65]
      xmax = 13,
      xmin = -13,
      ymax = 13, 
      ymin = -13, 
      grid = both]
      \addplot[
        domain=#1:#2,
        samples = #3,
        thick, red]{#4};\label{#5}
    \end{axis}
  \end{tikzpicture}

}


%%%
%
% Macroes
%
%%%

\newcommand{\sequence}[2]{
		\{ #1_1, #1_2, \dots, #1_#2 \}
}

\newcommand{\OS}{Operating System}

%insert code snippet
\NewDocumentCommand{\code}{v}{%
\texttt{#1}%
}

\title{\huge Systems and Networking Unit I}
\date{\small September 29, 2021 to \today}
\author{\large Dario Loi\\[6pt] \medskip \small Bachelor Degree in Applied Computer Science and Artificial Intelligence\\ \medskip  Sapienza University of Rome}

\begin{document}

\newcommand{\chapterquote}[3]{ % Quote, Author, Date %
	\begin{flushleft}
			\textit{
				``#1``
			}
		\end{flushleft}
		\begin{flushright} 
			--- #2, \textit{#3}
		\end{flushright}
}


\maketitle

\begin{abstract}
    
	In the following Hundred-or-so pages we will provide an 
	exhaustive set of notes pertaining to the lectures of 
	Operating Systems held by Professor Tolomei Gabriele 
	in the Winter Semester of 2021.

	The notes also represent my personal growth as a \LaTeX \ writer, 
	one can see that in the early chapters of this 
	document (and I am sorry for that) there are a lot of 
	sections and numberings in relation to the content, which 
	make the document a bit too noisy, while it is now unfeasible 
	for me to edit everything, I can at least say sorry in advance 
	to anyone reading this,

	I also like to think that as the document goes on the style 
	gets a little bit better and the functionalities that are 
	present increase.\\

	In this pages you will find an introduction to topics such as 
	Operating systems, Process Scheduling, Multithreading, Mass Storage 
	and File Systems, at the end of these notes a student will possess the 
	necessary theoretical knowledge to confront any oral examination 
	(or casual bar conversation, if you have \emph{that} kind of friends)
	on Operating Systems.\\
    
\end{abstract}

\tableofcontents


\VerbatimFootnotes

\chapter{Introduction To Operating Systems --- Lecture 1}

\chapterquote{Computers are like Old Testament Gods: Lots of rules and no mercy.}{Joseph Campbell}{21 Jun 1988}

	\section{Introduction}
	\subsection{Course Info}
	Firstly we begin with some general information pertaining the course

	The first unit of systems and networking focuses mainly on operating systems.

	Most of the material is going to be available on Moodle, the suggested book is
	\begin{quote}
		Modern Operating Systems --- Tanenbaum et al.
	\end{quote}


	\subsubsection{Exam Structure}

	The examination is divided in two phases:

	\begin{enumerate}
		\item Written Exam:
		The candidate answers 20 questions, getting 1.5 point for each correct one and losing 0.5 for each wrong one, a blank answer does not imply any change in score
		\item Oral Exam: 
		If one fail the Written with a mark between 15 and 17 he is also required to take an oral, the oral exam can also be taken by anyone who wishes to improve it's score.
	\end{enumerate}

	If we wanted to express this as a pseduo-logical formula, we would write something like:
	\[ score < 15 \implies \text{fail}; \; 15 \leq score \leq 17 \implies \text{oral}; \; score \geq 18 \implies \text{pass, oral for more score} \]

	\subsection{Course Structure}
	\begin{enumerate}[I.]
		\item Introduction
		\item Process Management
		\item Process Synchronization
		\item Memory Management
		\item Storage Management
		\item File System
		\item Advanced Topics
	\end{enumerate}

	\subsection{Term Glossary}
	\begin{itemize}
		\item OS = Operating System
		\item HW = Hardware
		\item SW = Software
		\item VM = Virtual Machine
		\item RR = Round Robin
		\item SJF = Shortest Job First 
		\item MLQ = MultiLevel Queue 
		\item MLFQ = MultiLevel Feedback Queue
		\item PCB = Process Control Block 
		\item TCB = Thread Control Block
	\end{itemize}


	\subsection{Overview of a Computer System}

	For our course we are going to take into account the basic
	model of a Von Neumann Architecture, containing a CPU processing
	information, a Memory (RAM) containing both data and instruction
	and some I/O modules allowing the interfacing of the machine with the outside world


	\section{Operating Systems}\label{sec:OS}
	\subsection{What is an OS}
	There is no unified definition for an OS, but a good enough one is:

	\begin{definition}[Operating System]

		Implementation of a Virtual Machine that is (hopefully)
		easier to program than bare hardware.

	\end{definition}

	The Operating system acts as a layer between the Application
	Programs and the hardware, using this encapsulation to
	provide ease-of-use and security, among many other benefits.

	\paragraph{What Is Inside an OS?} This question has no single answer
	since it is a design choice dependent on the requirement of the OS
	and the preference of the designer itself, we must therefore be satisfied
	with the knowledge that an OS must, in general, at least provide the
	properties we mentioned above

	\paragraph{Kernel and System Programs}\label{par:kernel}
	Kernel and System programs are two distinct sections of an OS,
	the kernel is an "always online" version of the Operating System, seen as the
	very core of the OS, all that is left outside of the kernel makes up the remaining
	system Programs

	\subsection{An OS's roles}

	\paragraph{A Referee:} An OS must manage resource conflict
	between programs in a fair way

	\paragraph{An illusionist:} the OS must have a proper abstraction
	of resource managing in order to mask the user from the clunkiness of
	memory management, and put up an illusion of infinite resources

	\paragraph{Glue:} Through the use of API it glues HW and SW together, and allows
	a clear communication between them

	\subsection{A Brief History of Operating Systems}

	We can divide the history of Operating Systems into different phases:

	\subsubsection{Phase I: Expensive HW, cheap Humans}
	\begin{enumerate}
		\item  At the beginning of the 50s
			transistors didn't exist, all of the programming was done straight
			on the machine, which were absolutely lacking of an OS, that is
			because the users of each computer were often the very academics who devised them

		\item 1955--1965, Mainframes: As time went one the computers expanded to allow
			the existence of a single mainframe sharing the resources with multiple terminals
			executing each job in sequence, the programs were written on Punched cards and therefore
			the OS was born, as a necessity to load these programs.

		\item 1955--1965 Batch Systems: another family of systems was Batch Systems, which
			utilized a primitive OS who used a pre-set scheduling to run a set of jobs in a single batch,
			this increased the throughput of the machine, while still requiring a substantial amount of
			manual intervention to make it work, since the scheduling was still not in the hands of the OS.

		\item 1955--1965 Multiprogramming Systems: a further step forward, the OS was now
			responsible for a multitude of things, jobs scheduling, memory and I/O management, the automation
			of these processes allowed the execution of multiple of them in parallel, thanks to context
			switching, this skyrocketed the efficiency of hardware, the only weak point being that
			the CPU was still stalled due to blocking I/O calls.
	\end{enumerate}
	\subsubsection{Phase II, cheap HW, Expensive Humans:}
	\begin{enumerate}
		\label{par:parallel}
		\item 1970-Current Times, Time Sharing: Many users are connected to the same machine, but
			the scheduling is managed by a time interrupt, this way the CPU is multiplexed between jobs.
			with a fast enough timer, we can actually obtain the illusion of parallelism
	\end{enumerate}

	\clearpage

	\subsubsection{Very Cheap HW, Very Expensive Humans}

	We enter the era of personal computers:

	\begin{enumerate}
		\item Initially these were developed with pretty simple OS, without multiprogramming,
			concurrency, memory protection, etc\dots

		\item Later on networking and GUI options were added to the OSs of PC, to better their
			user experience

		\item Finally, in our current times, PC are equipped with proper operating systems, such as:
		\begin{itemize}
			\item Windows NT (1991)
			\item Linux (1991)
			\item Mac OS X (2001)
		\end{itemize}

	\end{enumerate}

	\paragraph{The Current Times} there are now also a multitude of different environments in which an OS is required, thanks
	to the advancement in the IoT and the internet, everything is distributed and the computerization
	of many household appliances poses new requirements

	\paragraph{New Trends In OS Designs} to keep the pace with rapid HW development, and the paradigm
	shift that the new distributed systems has brought, there is a constant research on the development of more OSs
	capable to operate in different conditions

	\paragraph{Open Source}
	Thanks to the development of the Linux kernel, many of these niches can be filled with an Open
	source, custom-tailored implementation.

	\subsection{Why study OSs?}
	\paragraph{Learning CompSci Concepts} OSs represent perfectly the computer science concept of "abstraction", through
	their necessity to hide physical resources with virtualization

	\paragraph{System Design} OS design represent a multitude of choices that must be made, while taking into
	consideration the requirements of said OS, usually most of those choices present us with trade-offs: a fancier UX
	will come at the cost of Performance and Memory size of the OS; a greater number of layers of abstraction require an
	increased performance to be upheld, etc, etc\dots

\chapter{Computer Architectures --- Lesson 2}

\begin{center}
        
    \textit{\small This lecture contains lots of topics that have been approached in previous courses,
	for this reason, during this lecture many of the topics have been skipped, this \emph{does not} happen again 
	during the course of this document.}\\
	
	\textit{\small If anyone wishes to have a complete overview of the subject from the ground up I'd advise him to use this chapter 
	in conjunction with the course's.
	}
	    
    \end{center}

	\section{Lecture Outline}
	\begin{enumerate}[I.]
		\item Computer architecture review
		\item HW support for OS functionalities and services
		\item OS design
	\end{enumerate}
	\section{Computer Architecture}
		\subsection{Modern Computer Design Recap}

		Inside a modern PC design, all the main components are
		hooked up to a central BUS that handles their intercommunication:
		
	\begin{itemize}
		\item CPU (Central Processing Unit): he processor responsible for computation.
		\item Main Memory: Stores data and instruction used by the CPU.
		\item I/O Devices: Terminal, Keyboard, disks, etc\dots
		\item System Bus: The intercommunication infrastructure that 
		connects all the aforementioned components.
	\end{itemize}

		This architectural model is shared in-between a wide variety of devices, 
		from laptops to smartphones to workstations.
		
		\paragraph{Stored Program Philosophy} 
		this architecture is based on the Von-Neumann architecture, 
		allowing programs to be stored, instead of built in.

	\section{Buses and I/O}
		\subsection{The System Bus}
		At the beginning there was a single bus to handle all communication, in modern days the bus is split into three channels:
		
		\begin{multicols}{3}
			\begin{itemize}
				\item Data Bus
				\item Control Bus
				\item Address Bus
			\end{itemize}
		\end{multicols}
		
		
		\paragraph{I/O Devices} Every I/O device is split into two parts: The physical 
		device and the device controller, a specific chip or set of chip controlling the 
		device, every I/O Can be categorized as either a User Interface device or a storage device
		
		Each of these Controllers has a number of dedicated control registers to hold 
		flags relative to the status of the device, configuration registers that allow 
		the CPU to edit the device's behaviour, Data registers to act as buffers/cache for I/O
		
		\paragraph{Drivers} Drivers are pieces of software that enable an OS to interface with its peripherals
		
		\subsection{System Bus Addressing}
		\paragraph{Memory referencing} The CPU goes through a process for memory 
		referencing: it outputs the address of the memory on the address bus, 
		it raises a READ flag on the control bus.
		
		These two operations allow the memory to output the contents of memory at that address
		
		\paragraph{Switching Memory} The Bus chooses between memories thanks to a 
		special control line (M/\#I) that switches between main memory and storage I/O devices
		
		\paragraph{Device Control Philosophies} Port vs Memory-Mapped I/O are 
		two philosophies that allow the CPU to talk to a device controller, 
		previously Port Mapped was dominant, and employed a separate address space for every I/O device
		
		Memory-Mapping Allows controllers to address storage devices in the same space as main memory
		
		\paragraph{Port Mapping Instructions} are required in a Port Mapped Architecture, 
		the CPU must possess an IN and an OUT instruction to signal the operation to perform on the device. 
		
		When either of these two instructions are asserted, the M/\#IO line is negated, so the main memory does not respond
		
		\paragraph{Polling vs Interrupt} Polling and Interrupts are the two ways a 
		CPU can approach an I/O task, with polling, the CPU continuously checks on the I/O for status
		
		In an Interrupt based environment, the CPU receives an Interrupt when the 
		device is done, and can perform other tasks in the meanwhile
		
		\paragraph{Programmed or DMA} Before the CPU was responsible for the 
		movement of data in an I/O device.
		
		currently most devices interface with a DMA (Direct Memory Access) 
		controller, that acts as a secondary CPU to access data.
		
	\section{OS Services}
		\subsection{Protection and Security}
			\subsubsection{Permissions}
			\paragraph{Sensitive Instruction}
			A set of instruction, specifically 
			those related to interrupts, are "sensitive", meaning that they are 
			potentially harmful if executed by malicious users.
			
			The solution is to allow only a subset of users to execute these sensible instructions
			
			\paragraph{CPU Modes} to allow this differentiation of permissions, 
			we have two different CPU modes:
			
			\begin{itemize}
				\item Kernel Mode: The user can perform any instruction
				\item User Mode: the user is restricted, they can't:
				\begin{multicols}{2}
					\begin{itemize}
						\item Directly Address I/O
						\item Manipulate main memory
						\item Halt the machine
						\item Switch to Kernel Mode
						\item etc \dots
					\end{itemize}
				\end{multicols}
			\end{itemize}
			
			\paragraph{More Complex Solutions} Instead of just having two modes, 
			one can implement an n-ring protection tier system, allowing a different 
			range of privileges to different users, this can be implemented 
			in a size of $ log_2(n) $ bits:
			\subsubsection{Memory Protection}
				A modern architecture must provide support for the OS to protect 
				itself from other programs, preventing them to overwrite its 
				memory and each other's.
				
				\paragraph{A Simple Solution} is to have two dedicated registers, 
				a base and a limit register, containing the bounds of a process's memory. 
				these are loaded on program start-up by the OS.
				
		\subsection{System Calls and Exceptions}
			\subsubsection{System Calls: Preserving Usability} 
			Due to many of the operations being restricted to kernel mode, 
			the user mode is pretty limited in its usability, in order to preserve 
			the user's capability to perform certain tasks, we need a solution:
			
			\paragraph{System Calls} are used to ask the OS to perform a restricted 
			behaviour in a controlled manner for the user, these behaviours 
			include I/O operations and Network Transfers
			
			\paragraph{Exceptions} are used by the OS to handle undetermined 
			behaviour at runtime execution, such as division by zero
			
			\paragraph{Terminology} these are some general terms used while discussing system calls:
			\label{par:trap}
			\begin{itemize}
				\item Trap: a system call that "traps" the processor's execution, these 
				have several subtypes
				\begin{itemize}
					\item System Calls: called Software Traps (Synchronous),
					\item Exceptions: called Faults (Synchronous),
					\item Interrupts: (Asynchronous, since it is not called by the CPU).
				\end{itemize}
			\end{itemize}
			
		
		\subsection{OS-Programs Interface}
		\paragraph{System Calls}
			The operating system also acts as a hardware wrapper, allowing user programs 
			to interact with it only through System Calls 
			
			System Calls are written in a high level language, such as C++, most of the 
			time they aren't even directly provided by the OS, but through APIs.
			
			\paragraph{Process Control}
			Those include the end, abort, load, execute, create process, terminate process, 
			get/set, process attributes, sleep, throw or listen for event, malloc and free.

			\paragraph{File Management}
			Those include:
			\begin{itemize}
				\item File Creation,
				\item File Deletion,
				\item File Manipulation (open, close, read, write, get attributes).
			\end{itemize}

			File directory implementation is dependent on the way the File System is implemented 
			(See this .pdf's last chapter)

			\paragraph{Device Management}
			Those Incldue all operations regarding external I/O Devices such as disk or USB drives 
			and abstract devices such as partitions, RAM disks and even files in the case 
			that the system represents devices as a special kind of file\footnote{This is the 
			case in UNIX}.

			\paragraph{Information Maintenance}
			Those Include the calls to either get/set the time, date, system and 
			other device attributes, special system calls used to make a program 
			pause after each instruction and for tracing of program operations are 
			also included here\footnote{Such instructions are instrumental in the 
			creation of a \emph{debugger} for a programming language.}.

			\paragraph{Communication}
			Those include interrupts to create/delete connections between devices and 
			to send and/or receive messages through these connections, there are two 
			main methods with which this is achieved:

			\begin{itemize}
				\item Message Passing: Messages are passed in-between two processes,
				\item Shared Memory: Information/Data is exchanged in a shared section 
				of main memory, the exchange is regulated by locking mechanisms that restrict
				simultaneous access in order to prevent bugs.
			\end{itemize}


	
	\chapter{OS Design Goals --- Lecture 3}
			\section{System Calls Handling}
				every time that a system call is launched, it goes through a specific 
				pipeline that allows it to be performed:
				\begin{enumerate}
					\item The System Call gets called from a wrapper in an high level 
					language. (C/C++)
					\item The wrapper calls an Interrupt that points to a memory location 
					containing the generic System Call routine.
					\item The System Call gets handled by the System Call Handler, which 
					looks up the System Call's code on the System Call Table.
					\item The specific System Call subroutine selected by the user is executed. 
				\end{enumerate}
				
				Every step from number 2 to 4 is incapsulated in 
				\hyperref[par:kernel]{Kernel} mode, and allows 
				the system to specify the behaviour of the call without 
				user intervention.

			\section{Operating Systems Design Goals}
				\subsection{Design Goals}
					The internal structure of different of OS can vary widely, and is 
					dependent on a number of philosophical choices:

					\paragraph{OS Implementations} Early in history Operating Systems were
					implemented directly in assembly, providing higher efficiency
					but a very low portability, since it can only be executed on that 
					specific architecture 

					Today OSs are programmed in a variety of languages depending on the 
					specific part, with low level sections being programmed in assembly, 
					generic sections in C/C++
					and some high level sections even being done with scripting languages 
					such as PERL or Python 

					\paragraph{Different Philosophies, MS-DOS} MS-DOS has no modular 
					systems and no separation between user and kernel mode, meaning that 
					programs can access hardware directly.

					It is a super easy (relatively) system to implement, but very unsafe

					\paragraph{UNIX} UNIX utilizes a Monolithic Kernel, a huge piece of 
					software with every service living in the same address space, as if it 
					was one big process, most of modern OSs utilize this
					philosophy for their architecture

					The Monolithic Kernel is advantageous since having every subroutine 
					in the same address space means that to call each other they can just 
					use function calls, which speeds up
					inter-process communication immensely

					This provides an efficient but rigid structure 

					\paragraph{Layered Structure}
					\label{par:OS_layer}
					 The OS is divided into N layers, with HardWare on layer 0, 
					 with every layer using functionalities implemented in L-1 
					and exposing new functionalities to L+1

					This architecture is super modular and easy to debug, but provides a 
					lot of overhead from the encapsulation necessary for the layers
				
					\paragraph{Microkernel Structure} The opposite approach of the UNIX 
					philosophy is the Microkernel approach, containing just the basic 
					functionalities such as memory management, scheduling 
					and inter-process communication, the rest is handled in the user mode 
					part of the OS

					\paragraph{Monolithic vs Microkernel: Hybrid Trade-off} many modern 
					OSs use a mix of these two philosophies, like windows and apple being 
					mostly Monolithic but employing microkernel philosophies
					for certain subsections of their architecture 

				\subsection{Programs and Processes}	
					\label{ssec:proc}
					A program is an executable which resides on memory, 
					it contains the set of instructions needed to accomplish its job, 
					in an UNIX-like OS, it is an .exe file

					\paragraph{What is a Process}A process is a particular instance of a 
					program when loaded to main memory, in the UNIX example, 
					multiple .exe(s) can be executed at the same time

					the process is an OS abstraction, it is how the Operating System 
					sees the instance of a running program, in a process, instructions are
					executed sequentially by the CPU.

					\paragraph{How are Processes managed} processes are dynamic, this 
					means that they might enter different states, which must be tracked 
					by the OS, the OS must also initialize
					processes and allow the communication between them, we are next going 
					to see how the OS manages these processes.

				\subsection{Virtual Address Space Layout}
				The Virtual Address Space is a way for the OS to provide a contiguous 
				sequence of addresses to each process, each process receives the same amount of Virtual Address Space.

				The Virtual Address Space is an abstraction of physical memory, 
				since on practice, contiguous sequence of addresses are rare to come by, 
				the OSs maps physical addresses to virtual, contiguous ones to solve this problem.

				The range of Virtual Address Space that a process can generate is machine 
				dependent, for an example, in a 32-bit architecture the Virtual Address Space has range:
				\[ [\;0 \;,\; 2^{32} - 1\;]\]

				The Virtual Address Space is divided (in a common x86 Linux implementation) 
				as follows:
				\begin{itemize}
					\item {\makebox[2cm][l]{Text:} Contains exe Instructions}
					\item {\makebox[2cm][l]{Data:} Global and Static variables (initialized)}
					\item {\makebox[2cm][l]{BSS:} Global and static variables (uninitialized or initialized to 0)}
					\item {\makebox[2cm][l]{Stack:} Stack: a LIFO structure used to store data needed by function calls (\emph{Stack Frame})}
					\item {\makebox[2cm][l]{Heap:} Heap: used for Dynamic allocation}
				\end{itemize}

				\subsection{Stack Frame}
				\paragraph{What is a Stack} as explained before, a stack is a Last 
				In-First Out data structure that provides two built in functions, push and pop.

				\paragraph{Function calls} whenever a function is called, the scope's 
				variables are stored in a part of the stack, called stack frame, this is
				used to recover
				the scope's variables when the function returns, and to remember the 
				return address at which the program must resume its execution.

				the stack frame for each function is divided into three parts:
				\begin{enumerate}
					\item Function Parameters, return Address
					\item Back pointer to the previous stack frame 
					\item Local variables
				\end{enumerate}
				
				\begin{example}[Function Call]
					\begin{lstlisting}
	

	foo(a, b, c);

	converts to

	push c
	push b
	push a
	call foo
					
					\end{lstlisting}
				\end{example}

				\paragraph{Stack pointer} the stack pointer, called \%esp
				is responsible for pointing at each variable in the current stack frame 
				
				\marginnote{All of the specified registers are relative to the x86 architecture}

				\paragraph{Base pointer}due to the dynamic nature of the stack, we also keep 
				a base pointer \%ebp
				that points always to the beginning of the current stack frame

				\paragraph{Calling a function} when a function is called, the following 
				operations happen
				\begin{enumerate}
					\item The caller saves the current stack frame base pointer \%ebp to the stack
					\item The caller sets the new base frame pointer to the current value of \%esp, or the next free slot on the stack 
					\item The parameters are accessed through the \%ebp
					\item Local variables of the caller are accessed through the stack frame top pointer \%esp
					\item The old stack frame base pointer is restored 
				\end{enumerate}

\chapter{Processes (1 of 2) --- Lecture 4}
	\section{Process Management}
		\subsection{Process Execution States}
		\label{sec:procStates}
				At any given time, a process can be in any of the following 5 states:
				\marginnote{The Implementation of this is System-Dependent, but this generalization is common among OSs}
				\begin{enumerate}
					\item New: The OS has just set up the process.
					\item Ready: the process is ready to be executed, it is waiting to be scheduled on the CPU.
					\item Running: the process is actually executing instructions on the CPU.
					\item Waiting: the process is waiting for a resource to be available or an event to occur (I/O Interrupt, Disk Access, Timer, Etc\dots).
					\item Terminated: the process is either done working or terminated by the OS.
				\end{enumerate}

				
				\paragraph{Another look at System Calls} Most system calls (for example I/O ones) 
				are blocking calls. 

				This means that the caller can not do anything until the system call returns 
				while the OS schedules different, ready, processes on the CPU
				to avoid letting the CPU being idle. 
				
				Once the system call returns the blocked process is ready to be scheduled 
				for execution again,

				This means that in a multiprogramming system only the process that 
				requested the call is blocked, not the whole system!

				\paragraph{Process State} a process state consists of the following data:
				\begin{enumerate}
					\item The current program's code
					\item The current program's data 
					\item the program counter indicating the next instruction 
					\item CPU registers 
					\item the program's call chain (stack) and frame/stack pointers 
					\item the Heap size and heap pointer 
					\item the set of system resources in use
					\item the process's current execution state
				\end{enumerate}

				\paragraph{Process Control Block}\label{par:PCB}
				The main data structure used by the OS to keep track of a process is 
				the PCB, containing the process state, the PID or process ID,
				all the registers necessary to keep track of the state of a program, 
				CPU scheduling information such as priority.

				The PCB is created on process initialization and is stored in 
				\hyperref[par:kernel]{Kernel} memory, 
				since it is a protected data structure.

		\subsection{A more In-Depth view of Processes}
			\paragraph{Process Creation}\label{par:p_creation}
				Processes may create other processes through specific system calls, when this happens 
				the creator is called parent and the new process is called child, the parent process
				shares its resources and privileges to its children

				A parent can either wait for the child to complete or continue in parallel 

				Each process is given an integer identifier called PID

			\paragraph{UNIX process Creation} On a typical UNIX system the process scheduler is named \emph{sched}
			and is given PID 0, the first thing it does at system start-up time is to launch \emph{init}, which gives that process PID 1.

			init then launches al system daemons and user logins, and becomes the ultimate parent of all other processes.

			to create a process we can invoke the \code{fork()} system call. 

			\paragraph{Parent/Child Address Space} there are two possibilities for the address space of a children process relative to 
			the parent:
			\begin{enumerate}
				\item The child becomes an exact duplicate of the parent, sharing the 
				same program and data segment in memory,
				each process will have their unique \hyperref[par:PCB]{PCB}, including 
				program counter, registers and PID 

				this is the behaviour of the \emph{fork} system call in UNIX

				\item The child process loads a new program, with a new code and data segment.

				This is the behaviour of the windows \emph{spawn} system call, or \emph{exec} in UNIX as 
				a second step after forking.

			\end{enumerate}

			there are two options for the parent as well:
			\begin{enumerate}
				\item waiting for the child process to terminate by issuing a \emph{wait} system call. 
				\item Run concurrently with the child, continuing to process without being blocked. 
			\end{enumerate}

			\paragraph{New Process Creation}
			Whenever we log in into a UNIX machine, a shell process is started, from which 
			we can type commands.
			Every Command that we issue creates a new child process of shell, this process 
			implicitly makes use of two commands,
			fork and exec, in order to instantiate these new processes.

			\paragraph{Parallel Process Creation}
				If we wish to write an algorithm that instantiates multiple processes that 
				are child of a single one
				we can use a for loop:
				\begin{lstlisting}
for(const &n_processes : processes)
{
	if(is_child)
	{
		//execute child code 
		break
	}
	 
	//if here, you're in the parent 
	//fork a new child 

}
				\end{lstlisting}

\chapter{Processes (2 of 2) --- Lecture 5}
	\section{Process Termination}
			\label{sec:p_termination}
			\paragraph{Self Termination}
				Usually a process can issue a system call explicitly asking for its 
				termination 

				this typically requires the returning of an integer, which is 
				conventionally used to represent 
				the state in which the process terminated, for example if there was an exception or if it just finished 
				its tasks 

			\paragraph{OS Termination}
				Processes can also be terminated by the system, for a variety of reasons, for example a lack
				of resources or the inability of the process to respond 

				A parent might also kill its children if the task assigned is no longer needed, or for 
				example if the parent itself gets killed, and cannot therefore sustain its children anymore

			\paragraph{What Happens at Termination}
				When a process is terminated, all of its allocated resources are freed up, file 
				and I/O buffers are flushed and closed, etc. 

				the process's termination status and execution statistics are returned to the parent\footnote{
					In the case in which the process is an \emph{Orphan}, it is returned to \emph{init}
				}
				if he is waiting for the child to terminate

				Processes which are trying to terminate but cannot because the parent\footnote{
					As before, when the parent terminates, they become Orphans and get killed by
					\emph{Init}
				} is not waiting for them 
				are called \emph{Zombies}, this is a problem since if nobody takes responsibility for their 
				termination, this causes memory leaks until the parent is terminated

	\section{Scheduling}
	\label{sec:scheduling}
		\subsection{Process Scheduling}
			\paragraph{Scheduling Goals}
				the two main goals of process scheduling are:
				\begin{enumerate}
					\item Keep the CPU busy at all times, to make efficient use of the hardware.
					\item deliver acceptable response times for all programs, to keep the illusion 
					of multi programming alive.
				\end{enumerate}

				This is done by implementing a suitable algorithm for swapping processes 
				in and out of the CPU 

			\paragraph{Process State Queues}
				The OS maintains PCBSs of all processes in state queues, there is one 
				for each possible \hyperref[sec:procStates]{State} in the process's state diagram.

				There is also typically one queue of each I/O device.

				When the OS changes a process's status, it is popped from one Queue 
				and inserted into another, the way this is managed depends on the  
				policies implemented by the OS, which change based on the queue's 
				represented state.

		\subsection{Schedulers}
			\paragraph{Types of Schedulers}
				There are two types of scheduler implementation:
				\begin{itemize}
					\item \emph{Long-term Scheduler}: they run infrequently and are typically 
					implemented in systems that take very heavy computational loads.
					\item \emph{Short-term Scheduler}: runs very frequently, (about ever 100ms) 
					and must very quickly swap one process out of the CPU and swap in another one.
				\end{itemize}

				Some systems might also employ a medium-term Scheduler, which allows the 
				clearing of small jobs quickly to lessen system load in critical conditions.

		\subsection{Context Switching}
		\label{ssec:context}
			\paragraph{Context Switching}
				Context switching is the procedure with which the CPU suspends the current process and 
				runs one picked from the ready queue.

				This operation is highly costly, since we have to state all the current states of the
				outgoing process to its \hyperref[par:PCB]{PCB}, in order to resume its execution seamlessly later

				a context switch happens whenever there is an incoming \hyperref[par:trap]{trap}

			\paragraph{Context Switch Fairness}
			\label{par:times}
				\marginnote{\emph{Time Slice} is the maximum amount of time between two context switches.}
				I/O bound processes get frequently switched due to I/O requests, while CPU-bound processes
				can run for much longer, theoretically forever if they never issue any I/O request

				To avoid this, context switching can also be triggered via a HW timer interrupt at any time slice,
				this ensures that there is a maximum time that bounds process execution, this not only improves
				fairness in process scheduling, but also improves the illusion of parallelism, since the 
				system can execute processes sequentially while switching them very fast, it looks like they are
				executed in parallel.

			\paragraph{Context Switch Frequency}
			\label{par:t_perf}
				One must carefully balance the size of the time slice, since every time we context switch 
				we pay a significant overhead which just throws away a slice of our computation power 
				
				\begin{itemize}
					\item For a smaller slice time we have a higher responsiveness, but we pay more overhead.
					\item For a larger slice we have less responsiveness but we minimize the overhead.
				\end{itemize}

				in modern systems time slices are between 10 and 100ms, while context switching takes
				around $10\mu s$, so the overhead is acceptably small in relation to the time slice.

	\section{Inter-process Communication}
			\paragraph{Inter-dependence or Cooperation}
				Processes can either be independent of cooperating, independent processes run
				concurrently with others in a multiprogramming OS, but can neither affect nor
				be affected by other processer. 
				Cooperating processes conversely, can and will affect or be affected by other
				processes in order to better achieve a common task:


			\paragraph{Benefits of Process Cooperation}
				\begin{itemize}
					\item Information sharing: cooperation allows the sharing of memory resources 
					\item Computation speedup: by breaking the task into subtasks some problems can be solved faster
					\item Modularity: modern architectures love to split up a system into many cooperating modules that 
					manage its different aspects\footnote{This is very common in GUI applications that employ an MVC model.}
					\item Convenience: it allows for a better illusion of parallelism, by allowing execution of a single program 
					in multiple windows which split system resources. 
 				\end{itemize}

			\paragraph{Shared Memory vs Message Passing}
				\emph{Shared memory} is much faster once the area is allocated, since it allows to bypass system calling,
				it is however complicated to program safely and is usually not very portable across architectures

				it is the preferable alternative when a large amount of data must be shared across processes on
				the same computer \\

				\emph{Message Passing} is slower since it must go through the kernel through system calls, but 
				allows safety and portability through system calls encapsulation

				It is preferred when frequency or amount of data is small, or when the transfer happens through
				multiple hosts, in which case not only there might not be a memory to share, but the transfer 
				might need to be performed over the net through the use of protocols such as 
				TCP or UDP.

	\section{CPU Short-Term Scheduling: an In Depth View}
			\paragraph{Our objectives}
			we wish to provide knowledge on the following topics
			\begin{itemize}
				\item Basic Scheduling Topics
				\item Common Scheduling Algorithms
				\item Scheduling Criteria/Metrics
				\item Advanced Scheduling concepts
			\end{itemize}


		\subsection{Basic Concepts}
			Almost every program has an alternating sequence of cycles dedicated 
			to ALU operations and cycles which wait for I/O

			even a simple fetch from memory takes orders of magnitude more than an
			arithmetic operation\footnote{This is a major problem in computer architecture, called the 
			\href{https://en.wikipedia.org/wiki/Von_Neumann_architecture\#Von_Neumann_bottleneck}{Von Neumann Bottleneck},
			which is a result of the homonymous Von Neumann Architecture in use in all of our current computers.
			}

			We wish to solve this through our scheduling algorithm 
			
			\paragraph{Maximizing Execution Time}
				in a system where a single process is run, the cycles spent waiting for 
				I/O are just lost forever, luckily, in a modern system, we are most likely 
				working in a multiprogramming environment, therefore we have a solution:

				We can utilize our scheduler to temporarily context switch another process 
				while our current one is put in an I/O waiting queue, allowing us to waste 
				less cycles while we wait

			\paragraph{CPU and I/O burst cycles}
				Due to this philosophy, we can divide any process's runtime into two 
				cyclical states, CPU and I/O burst, in each, the process is either 
				performing CPU computations or it is waiting for an I/O interrupt.

				Interestingly, bursts seems to be statistically distributed in a skewed 
				manner, where the majority of bursts last a very short time\footnote{Around 8ms.},
				while only a smaller part of them takes a very significant time.

				This in turn means that a majority of the waits is well-handled by modern solutions 
				to the Von Neumann bottleneck, while only few slip through the optimizations\footnote{One such situation
				could be a memory fetch that misses in all caches in memory hierarchy,and therefore needing a huge waiting time
				to fetch data directly from the disk} and actually take a significant amount of waiting time.

		\subsection{CPU Scheduler Implementation}
			Whenever the CPU enters an idle state, it picks another process from the \emph{ready} queue\footnote{
				Although we use the term queue, the queue itself is not necessarily implemented through a FIFO data structure.
			} to execute them.

			to recap, CPU scheduling happens when one of these four condition verifies:
			\begin{enumerate}
				\item When a process switches from running state to waiting state. (if it requests an interrupt)
				\item When a process switches from running state to ready state. (if it has executed past its time slice)
				\item When a process switches from waiting state to ready state. (if its interrupt has completed)
				\item When a process is either \hyperref[par:p_creation]{created} or 
				\hyperref[sec:p_termination]{terminated} by the system (or by itself/a parent).
			\end{enumerate}
			
			\paragraph{Non-pre-emptive vs Pre-emptive}
				\label{par:preem}
				Non-pre-emptive scheduling takes place only when 
				there is no choice, (in conditions 1 and 4)
				once a process starts it runs until it either blocks voluntarily or it finishes, without interference
				by the scheduler.

				Pre-emptive scheduling happens whenever scheduling takes place in all of the four aforementioned conditions 

				Modern systems make extensive use of pre-emptive scheduling, since it allows for superior 
				control by the OS, preventing processes from hogging the CPU.


			\paragraph{Pre-emption problems}
				Pre-emption causes troubles if it occurs while either the kernel is implementing a system call or 
				two processes are sharing data and they get interrupted in the middle of an atomic operation being
				performed upon these shared data structures.

				the possible solutions are:
				\begin{itemize}
					\item Make the process wait until the systems calls are either completed or blocked before  
					allowing pre-emption\footnote{This is problematic since it casts away the illusion of parallelism.}
					\item Disabling interrupts before entering critical code and re enabling 
					them immediately afterwards. \footnote{This must be implemented carefully
					lest it results in a \href{https://en.wikipedia.org/wiki/Halt_and_Catch_Fire_(computing)}{Halt and Catch Fire} situation.}
				\end{itemize}
				
\chapter{Scheduling Algorithms --- Lecture 6}
		\section{The Dispatcher}
			The dispatcher is the software module that gives control of the CPU to the 
			process that was selected by the scheduler, its functions include:
			\begin{itemize}
				\item \hyperref[ssec:context]{Context Switching}. 
				\item Switching to User-Mode.
				\item Jump to the program counter location of the newly loaded program.
			\end{itemize}

		The dispatcher runs on every context switch, therefore the dispatching delay
		must be as short as possible 
		\section{Dispatcher Design Choices}
			\subsection{Useful KPI for performance measurement}
				\label{ssec:KPI}
				the following KPI\footnote{Key Performance Indicator.} are used to measure Dispatcher/Scheduler
				performance
				
				\begin{enumerate}
					\item Arrival Time: time it takes a process to arrive in the ready queue
					\item Completion Time: time to complete execution 
					\item Burst Time: time required for CPU execution  
					\item Turnaround Time: difference between completion and arrival
					\footnote{That takes into account every state that the process
					is in, from first entering the ready queue to completion.}
					\item Waiting Time: difference between Turnaround and Burst Time\footnote{Takes into account
					the total time spent in the waiting queue, waiting for outside events.}
				\end{enumerate}

				we also consider this set of criteria and metrics to optimize in order to 
				optimize the scheduler as a whole:

				\begin{itemize}
					\item CPU Utilization: It should be close to 100\% in order to maximize 
					CPU usage
					\item Throughput: We would like as many processes completed in a certain 
					$\Delta{t}$
					\item Turnaround Time: We would like to minimize the time that a process 
					spends from start to completion.
					\item Waiting Time: It should be as small as possible, to minimize the time
					it takes for a process to be picked.
					\item Response Time: The time it takes from the issuance of a command to 
					its execution (key metric for the illusion of parallelism)
				\end{itemize}

				Ideally we would want a scheduler that optimizes all those metrics and KPI, but 
				generally this is impossible, since for some optimizations, some metrics are 
				opposite, and therefore we must accept some trade-offs.\footnote{As stated before, for 
				increased responsiveness we trade performance away.}
				
			\subsection{Dispatcher Policies}
				To choose what metrics to optimize and which to ignore, we can stick 
				by a design policy, depending on our requirements
			\begin{itemize}
				\item Improve System Responsiveness
				\begin{itemize}
					\item Minimize average response time.
					\item Minimize the maximum response time.
					\item Minimize variance of response time.
				\end{itemize}
				\item Improve System Performance
				\begin{itemize}
					\item Maximize Throughput (by minimizing Context switch overhead)
					\item Minimize Waiting Time (by enforcing a strict time slice policy)
				\end{itemize}
			\end{itemize}

			\paragraph{Initial Assumptions}To define the implementation of these policies, we take some assumptions:
			We assume one process per user, independent processes and single thread 
			per process.

			This is the simplest case to take into account and makes it very easy to provide a solution.

	\section{Scheduling Algorithms}
	\label{sec:scheduling_alg}
		We now take a look at a set of scheduling algorithms that can be used to solve 
		the scheduling problem 
		\subsection{First Come First Serve (FCFS)}
			\label{ssec:FCFS}
			The simplest implementation, the ready queue is indeed treated as a 
			FIFO queue, like a line at the post office.

			The scheduler executes jobs in a \hyperref[par:preem]{non-pre-emptive} fashion, switching 
			only when a job is completed.

			\paragraph{Waiting time Calculation}
				\begin{align*}
					&N = \text{Number of jobs} \\
					&t^{arrival}_i \; = \; \text{time for a process to get in the queue}\\
					&t^{completion}_i \; = \; \text{time for a process to finish running} \\
					&t^{burst}_i \; = \; \text{total time that the process spends executing ALU operations} \\
					&t^{turnaround}_i = t^{completion}_i - t^{arrival}_i \\
					&t^{waiting}_i = \frac{1}{N}\sum^N_{i=0}{t^{turnaround}_i - t^{burst}_i}
				\end{align*}

			From our analysis the waiting time is minimized if the input is sorted in increasing
			order of CPU burst time, since every process must then wait the minimal
			amount of time for execution. 

			\begin{itemize}
				\item Pro: Very Simple to implement!
				\item Con(s): High average variance, Convoy effect (CPU bound jobs forces I/O to wait)
			\end{itemize}

		\subsection{Round Robin (RR)}
			\label{ssec:RR}
			Round Robin operates in a similar fashion to FCFS, the only difference is that
			we implement the \emph{time quantum} (or \hyperref[par:times]{Time slice} )
			solution in order to implements pre-emptiveness.

			If a job finishes before the time quantum expires, it is swapped out of the CPU 
			like a normal FCFS, if the timer goes off, the jobs gets put in the back of the 
			ready queue, implementing a sort of "circular" queueing.

			\paragraph{Feasibility}
			this solution starts being feasible to implement in a primitive modern system,
			it gives a fair share to each Job, but can increase average waiting times due 
			to this fairness, in comparison to other, more refined algorithms.

			\paragraph{Time Slice Size Sensibility} 
			The performance of \hyperref[ssec:RR]{RR} depends on the 
			selected time quantum, a too large time quantum just degenerates to FCFS,
			as jobs are never pre-empted by from the CPU, skyrocketing the waiting time;
			if the time quantum is instead too large, we experience a loss in throughput,
			as we \hyperref[par:t_perf]{previously explored}

			\paragraph{Upper Bound Computation}
			We can compute the upper bound on Waiting time of a process before 
			execution is resumed as follows:
			\begin{align*}
				& N = \text{Number of jobs}\\
				& \delta = \text{time slice} \\
				& sup\{ t_i^{start}\} = \delta ( i - 1), \forall i \in \{1, 2, \dots, N\}
			\end{align*}

			As we can see the worst-case scenario is that every process takes the full
			execution time and only gets interrupted by running out of its time quantum.

			\paragraph{Pros and Cons}
			for a RR approach, we have:
			\begin{itemize}
				\item PRO(s): Very low variance, the system is much more reliable in 
				keeping up a better illusion of \hyperref[par:parallel]{parallelism}.
				\item CON(s): slower turnaround and waiting time compared to FCFS, even
				without \hyperref[ssec:context]{context switching} overhead.
			\end{itemize}
		
		\subsection{Shortest Job First (SJF)}
		\label{ssec:SJF}
			Another idea for a policy is to schedule the job that has the least 
			\emph{expected} amount of work to do until its next I/O operation 
			or termination, by amount of work we mean CPU burst duration.

			The reason for this is to provide an algorithm with similar functionality to 
			\hyperref[ssec:FCFS]{FCFS}, but to force it to always execute in the best case
			possible (the CPU burst times are sorted in increasing order)

			\paragraph{Pros and Cons}
			or a SJF approach, we have:
			\begin{itemize}
				\item PRO(s):
				\begin{itemize}
					\item Provably Optimal when minimizing avg. waiting time
					\item Can be made to work both with pre-emptive and non-pre-emptive 
					schedulers
				\end{itemize}
				\item CON(s)
				\begin{itemize}
					\item Estimating CPU burst duration is incredibly hard to predict.
					\item CPU-Bound process are starved, since I/O bound processes have,
					in average, a low CPU burst duration.
				\end{itemize}
			\end{itemize}
		
			\subsubsection{Predicting CPU Burst Time: Exponential Smoothing}
				A statistical approach to predicting burst time based on historical
				measurements of recent burst time is as follows:

				\begin{align*}
					x_t &= \text{Actual length of the t-th CPU burst} \\
					s_{t+1} &= \text{predicted length of the (t+1)-th CPU Burst} \\ 
					\alpha \in \mathbb{R}&, \; 0 \leq \alpha \leq 1 \\
					s_1 &= x_0 \\
					s_{t+1} &= \alpha x_t + (1-\alpha) s_t 
				\end{align*}

				We have the following edge cases:
				\begin{align}
					\alpha &= 0 \implies s_{t+1} = s_t, \\
					\alpha &= 1 \implies s_{t+1} = x_t
				\end{align}

		\subsection{SRTF (Shortest Remaining Time First): A pre-emptive Approach to SJF}
			\label{ssec:SRTF}
			SRTF (Shortest Remaining Time First) Executes in a similar manner to
			\hyperref[ssec:SJF]{SJF}:

			The shortest job is executed first and keeps executing until a job with a 
			shorter remaining time enters the ready queue, in which case a context 
			switch is performed and the shortest job is put into execution. 

			\paragraph{Optimality}
				If we analyse average waiting and turnaround times, we can check that 
				this implementation returns the shortest times when compared with other
				algorithms (it is optimal)

	\section{Priority Scheduling}
		As seen before, we can use some useful metric to sort processes in a manner that 
		allows our system to increase its efficiency; we can now observe that we can 
		choose whatever metric to utilize as a priority, allowing us to maximize different
		\hyperref[ssec:KPI]{KPIs} depending on our policy. \\

		This spawns an array of different scheduling algorithms that make use of a
		priority system to define different approaches with which to solve the scheduling 
		problem\footnote{\hyperref[ssec:SJF]{SJF} is one of such algorithms, even if it is 
		possibly one of the most primitive implementations of such a system.}.
				
		\subsection{Priority Scheduling Characteristics}
		In a priority scheduling algorithm, priorities can be assigned either 
		\emph{internally} or \emph{externally}

		\begin{itemize}
			\item Internal priorities are assigned by the OS using different metrics and 
			KPIs depending on the desired policy.
			\item External priorities are assigned by users based on the importance of the
			job to compute, that can be determined by a number of empirical factors.
		\end{itemize}

		Priority scheduling can also be implemented in a pre-emptive or non-pre-emptive
		\footnote{See: \hyperref[par:preem]{Pre-emptiveness}.} Fashion.

		\paragraph{Common Priority Scheduling Issues}
		When jobs are executed according to a set of priority rules, a number of problems
		may arise, for example they can encounter Indefinite Blocking, also known as 
		\emph{Starvation}, which occurs whenever a low-priority job waits forever due to 
		the presence of higher priority jobs in the queue which never get completed.

		A solution to the Starvation problem is implementing a process called \emph{Aging},
		in which processes get their priority bumped up the more time they wait, so that 
		they are eventually scheduled. \\ 

		We now see a number of priority based Implementations:


	
	\subsection{MLQ (Multi-Level Queue)}
	\label{ssec:MLQ}
	We explore the idea of defining different categories\footnote{For example: CPU-Bound
	and I/O Bound jobs can be separated into different categories.} of jobs we can then implement
	a queue for each category and an algorithm for each queue

	\paragraph{Meta-Scheduling} 
		This poses an interesting conundrum: We must then choose which queue to pick from,
		therefore implementing a kind of meta-scheduling algorithm.

		Two common choices are:
		\begin{itemize}
			\item Strict Priority: No job in a low priority queues runs until all high
			priority ones are empty
			\item Round Robin: Jobs are executed in a \hyperref[ssec:RR]{Round Robin} fashion,
			with a specific time slice\footnote{This allows us to still provide different
			priorities in a more fair manner.} for each queue.
		\end{itemize}

	\paragraph{MLQ Hierarchy}
	\label{par:MLQ_h}
		In general, in a \hyperref[par:OS_layer]{layer-like} Operating System,
		system processes have a higher priority and shorter time slices, since they are
		generally I/O Bound processes, whereas user processes have a lower priority and a 
		longer time slice, since they are usually CPU bound and make use of I/O through 
		system calls. 


	\subsection{Multi-Level Feedback Queue (MLFQ)}
	\label{ssec:MLFQ}
		MLFQ operates in a similar way to \hyperref[ssec:MLQ]{MLQ}, with the difference
		that jobs can be moved between queues, this is required when:
		\begin{itemize}
			\item A job changes between CPU-bound and I/O-bound
			\item A job has waited or a long time and can be bound to a higher priority
			queue (to compensate for aging)
		\end{itemize}

		\subsubsection{MLFQ Implementation}
		Jobs first start in the highest priority queue, if its time quantum expires,
		it gets dropped by one priority level\footnote{This indicates that this job
		is CPU-Bound.}.

		If the time slice does not expire, but a context switch still occurs due to an 
		interrupt, the priority is increased by one level\footnote{This indicates that 
		the job is I/O-Bound.}.

		Therefore this system quickly separates I/O and CPU bound jobs, allowing the 
		\hyperref[par:MLQ_h]{hierarchy} to impose itself naturally.

		\paragraph{MLFQ Advantages and Disadvantages}
		This system is the most flexible (since it can implement all the other algorithms
		in its queues), but it's very complex to implement, since it is defined by many 
		parameters:
		\begin{itemize}
			\item The number of queues.
			\item The scheduling algorithm for each one.
			\item The rules for promoting or demoting processes.
			\item The default queue(s) for a new process.
		\end{itemize}

\chapter{Scheduling and Threads --- Lecture 7--8}
	\section{A Last Look at Scheduling}
		\subsection{MLFQ: A more in depth view}

			\paragraph{MLFQ Design Philosophy}
			MLFQ allows for a high degree of flexibility, it is in fact the most flexible of
			the scheduling algorithms, it is also very efficient since it aims to mimic the 
			optimality of \hyperref{ssec:SJF}{SJF} 
			without its common pitfalls, mainly, getting slowed down by
			short, I/O bound jobs.

			\paragraph{MLFQ Flexibility}
			MLFQ prevents this by promoting CPU bound jobs to a higher level, so they can be 
			treated in a specific queue\footnote{As stated before, there's a high probability
			the high priority queue is running \hyperref[ssec:SJF]{SJF}.}, separating them from the 
			problematic I/O bound jobs, that get instead demoted to other queues that can handle 
			them more accordingly.

			\paragraph{MLFQ Fairness}
			Through the aforementioned principles, MLFQ achieves fairness by:
			\begin{itemize}
				\item Giving each queue a fraction of CPU time.
				\item Adjusting priorities Dynamically, thus avoiding starvation.
			\end{itemize}

		\subsection{Lottery Scheduling (LS)}
			\label{ssec:lottery}
			\paragraph{Implementation}
			We distribute a certain number of \emph{lottery tickets} among the jobs to be scheduled,
			on each time slice we pick a winning ticket with a uniform random probability.

			We expect the CPU time to be assigned in a similar fashion to the original distribution 
			of tickets, due to the nature of uniform distribution when stretched over infinite fractions
			of time.


			\paragraph{Efficient Distributions}
			A valid policy is to assign tickets as follows:
			\begin{multicols}{2}
				\begin{itemize}
					\item Give more tickets to shorter jobs 
					\item Give less tickets to longer jobs
				\end{itemize}
			\end{multicols}
			

			this is interpretable as a probabilistic approach that aims to approximate the 
			behaviour of
			\hyperref[ssec:SJF]{SJF}, to avoid the problem of starvation, we give each job at
			least a ticket, this also leads to an interesting behaviour: 
			\begin{center}
				As the jobs decrease, the probability that the remaining jobs get picked increases
			\end{center}

			\paragraph{CPU Assignments}
			\begin{align*}
				&n_{short} = \text{total number of short jobs}\\
				&n_{long} = \text{total number of long jobs}\\
				&N = n_{short} + n_{long} = \text{total number of jobs}\\
				&\;\\
				&m_{short} = \text{number of tickets assigned to each \emph{short} job}\\
				&m_{long} = \text{number of tickets assigned to each \emph{long} job}\\
				&\;\\
				&M = \begin{pmatrix}m_{short} \\ m_{long}\end{pmatrix} \cdot \begin{pmatrix}n_{short} \\ n_{long}\end{pmatrix} = m_{short} \cdot n_{short} + m_{long} \cdot n_{long}\\
				&\;\\
				&CPU_{short} = \frac{m_{short}}{M}\\
				&CPU_{long} = \frac{m_{long}}{M}
			\end{align*}

			

	\section{Scheduling Algorithms Summary}
	We take into consideration the general Characteristics of all the 
	algorithms we explored lately:
	\begin{itemize}
		\item \hyperref[ssec:FCFS]{FCFS} Very simple, non pre-emptive, generally unfair
		\item \hyperref[ssec:RR]{RR}: Also Very Simple, most fair, high waiting time
		\item \hyperref[ssec:SJF]{SJF}: Generally unfair, risk of starvation, provably
		optimal in terms of average waiting time, if we are able to accurately predict
		the next CPU burst time.
		\item (\hyperref[ssec:MLFQ]{MLFQ}/\hyperref[ssec:MLQ]{MLQ}): Approximates SJF
		while retaining a high level of flexibility, solving some of its shortcomings.
		\item \hyperref[ssec:lottery]{Lottery Scheduling}: Mostly fair implementation,
		with a very predictable nature due to its very own implementation.
		
	\end{itemize}
	
	\section{Threads}
	\label{sec:threads}
		\subsection{Beyond Single Threaded processes}
		Generic Process Models are implemented by executing instructions 
		sequentially one after another, this means that in that specific 
		process there is only one \emph{single thread} of control, this 
		isn't the modern approach anymore:

			\paragraph{Multi-Threading}
			All Modern Operating Systems provide features enabling a process 
			to contain \emph{multiple threads} of control, we now want to
			introduce a set of concepts necessary to develop multi-threaded 
			computer systems, while also taking a look at the number of issues 
			that arise when utilizing these architectures, and their effect on 
			the design of the \hyperref[sec:OS]{Operating System} as a whole.

		\subsection{Threads: An Overview}
			\paragraph{What is a Thread}
			A thread is a basic unit of CPU utilization, it consists of a 
			program counter, a stack, a set of registers, and an ID to 
			uniquely identify the thread.

			Therefore multi-threaded applications possess multiple threads 
			sharing a single process's resources, each possessing the unique 
			characteristics enumerated above. 

			\paragraph{Process vs Thread}
			the confusion between process and thread can easily arise, here 
			are some distinctions between them:
			\begin{itemize}
				\item A \hyperref[ssec:proc]{Process}: defines the address space, data and resources
				sharing them between threads, it requires system calls to 
				communicate with other processes.
				\item A Thread: defines a sequential execution stream within
				a process, it is therefore bound to its parent process, this 
				brings a set of benefits to it: communication with other 
				children of the same process is easier, since they all share 
				the same address space and therefore require no system calls. 
			\end{itemize}

			\paragraph{The Motivation Behind Threads}
			Threads are an additional layer of abstraction that builds upon processes,
			as usual, this implies a necessary overhead in order to implement the abstraction
			itself, this begs the question:

			\begin{center}
				Why do we need Threads? why is the additional performance burden worth it?
			\end{center}

			The answer is, threads are exceptional in situations where a single program 
			must perform CPU-Bound tasks that are different in nature, and that therefore
			can execute in parallel, effectively allowing us to perform computation in one 
			task when the other could be blocked\footnote{This provided that both tasks are
			not utilizing a shared resource.}.

			This gifts us with an increase in performance that is often measured in orders 
			of magnitude, that is, if the task is \emph{easily parallelizable}, meaning that 
			we do not have to access shared resources very often or have threads wait for
			each-other's completion frequently, as both of those cases often lead to a return
			to a Singe-Threaded style of execution, with the added abstraction overhead on top,
			which in turn has the effect of lowering our effective performance.

			\paragraph{Why Not Processes}
			Another, similar, question might arise:
			\begin{center}
				Why do we need Threads? why can't we just spawn another process?
			\end{center}

			Theoretically, this is perfectly reasonable, as we could instantiate
			a new single threaded process for each thread we want to use, but we
			have at least two reasons why this is not the right choice:
			\begin{enumerate}
				\item Inter-Thread communication is significantly faster due to
				shared memory space
				\item \hyperref[ssec:context]{Context Switching} is significantly faster between threads 
				spawned from the same process, due to shared Information
			\end{enumerate}

			These two reasons are more than enough to justify the utilization
			of threads.

			\paragraph{Common Utilizations and Benefits}
			Threads are commonly found in these applications:
			\begin{itemize}
				\item Web Servers: to handle multiple requests at once.
				\item GUI Applications: to handle various parts of the GUI for increased 
				responsiveness.
				\item Videogames: Parallelization allows separate calculations for 
				independent areas such as physics and graphics, improving framerate.
			\end{itemize}

			This is because, given that the algorithm or task to parallelize respects the 
			previously stated criteria and therefore benefits from multithreading, it 
			gains the following advantages:

			\begin{itemize}
				\item Responsiveness: Dedicating specific threads to areas of the programs 
				that handle I/O means that the program is always ready to listen for user 
				Input.
				\item Resource Sharing: a Multi-threaded program shares
				the same address space, which allows for faster context switching 
				\item Economy: For reasons stated before, thread management (not only
				context switching, but also creation and deletion) is much faster than
				process management.
				\item Scalability: Multi-threaded programs can make use of all the system's 
				resources and can lend themselves to a distributed architecture more easily,
				allowing the computation to be performed on a multitude of systems, dramatically 
				increasing performance.
			\end{itemize}

			This loads of benefits have pushed recent Computer Architecture designs towards 
			multi-core designs, having multiple CPUs on a single chip, allowing for true 
			parallel processing, this is also because we recently made a push for less power 
			consumption in CPUs, and statistically the efficiency in performance gain
			of multi-coring is higher than that of increasing the number of transistors per core, 
			meaning that the ratio $\frac{IPC}{Watt}$ is higher.

		\subsection{Different Types of Parallelism}
		There are multiple ways of parallelizing workloads in a multithreaded architecture,
		we can identify two:
		\begin{enumerate}
			\item Data Parallelism
			\item Task Parallelism
		\end{enumerate}

		Let us discuss these in detail further on.

			\paragraph{Data Parallelism}
			Data Parallelism consists of dividing the data up in equal chunks, and assigning
			each chunk to a different thread, that performs parallel computation on the Data 
			on a different core.\\

			This is Especially efficient in \href{
				https://en.wikipedia.org/wiki/Embarrassingly_parallel
				}{
					Embarrassingly Parallel
			} problems, for example a GPU computing the next frame on the monitor splits
			the frame buffer into equally sized chunks of a certain size (e.g $16\times16$)
			and makes use of its many cores (usually in the hundreds) to draw each pixel 
			independently from the other. 


			\paragraph{Task Parallelism}
			Task Parallelism consists in individuating different, independent areas in our 
			program and assigning each on a thread, this is particularly useful in complex,
			user driven applications where most functionalities depend on user input.

		\subsection{OS Classification Based On Thread Management}

		Operating Systems can be classified based on the way that they manage multi threading 
		in relation to address spaces, for example:

		\begin{enumerate}
			\item Single Address, Single Thread: 
			These include all archaic operating systems, a prime Example is MS-DOS
			\item Multiple Addresses, Single Thread: 
			All operating systems that support multiprogramming while not yet having the 
			definition of thread implemented, for example UNIX
			\item Single Address, Multiple Thread:
			A peculiar kind of Operating Systems that only allow for a single address space 
			in which multiple threads can live, for example, Xerox Pilot
			\item Multiple Addresses, Multiple Threads:
			All modern Operating Systems which support both Multiprogramming and multithreading,
			such as Windows NT, Mach and Solaris.
		\end{enumerate}

		\paragraph{OS Thread Support}
		Support for multithreading can be provided in both Kernel and User level, with threads 
		taking names of respectively Kernel or User Threads

	\section{Kernel and User Threads}
	Programs can implement multithreading either in \emph{User} space or in \emph{Kernel}
	space, threads have different behaviour depending in which space they are implemented:

	\paragraph{Kernel Threads}
	Threads living in the kernel represent the smallest unit of execution schedulable 
		by a modern OS, these are fully-fledged threads in the system's kernel memory, and 
		therefore posses a PCB and a TCB (Thread Control Block), they have the following characteristics:
		\begin{itemize}
			\item PRO(s):
			\begin{itemize}
				\item The kernel has full knowledge of all threads.
				\item The scheduler can decide to give more CPU time to a process that 
				has a higher number of threads.
				\item It is optimal for applications that block frequently.
			\end{itemize}
			\item CON(s):
			\begin{itemize}
				\item Significant overhead and increase of kernel complexity.
				\item Slow and Inefficient.
				\item Context switching managed by the kernel.
			\end{itemize}
		\end{itemize}

	\paragraph{User Kernels}
	Threads living in User Space are managed entirely by the run-time system provided 
	by the user level \emph{thread library} \footnote{Such as the \texttt{<thread>} library in C++.},
	the operating system has no knowledge of the threads evene existing and therefore manages 
	user-level threads as if they were a single-threaded \emph{process}.

	\begin{itemize}
		\item PRO(s):
		\begin{itemize}
			\item Fast and Lightweight.
			\item Scheduling is more flexible.
			\item No SysCall required.
			\item No Context-Switching
		\end{itemize}
		\item CON(s):
		\begin{itemize}
			\item No \emph{true} concurrency.
			\item Poor scheduling decisions.
			\item Lack of coordination betwen kernel and threads.
			\item Requires non-blocking system calls, otherwise all threads within a process 
			have to wait.
		\end{itemize}
	\end{itemize}

\section{Thread Management Models}
The mapping of user threads to kernel threads can vary depending on implementation, such as:
	\begin{enumerate}
		\item Many-to-One.
		\item One-to-One.
		\item Many-to-Many.
		\item Two-Level.
	\end{enumerate}

	\paragraph{Many-to-One}
	Many user threads are mapped onto a single kernel thread, this effectively means that 
	the process can only run one user thread at a time since there is only one kernel thread 
	associated to it\footnote{remember that kernel threads are the \emph{smallest} unit of 
	execution that an OS can schedule.}.

	a Many-to-One solution is terrible for multi-core systems since it means that 
	no actual concurrency is occurring, it is also vulnerable to blocking system calls since 
	a single call can block the entire process.

	\paragraph{One-to-One}
	One user thread is mapped onto one kernel thread, this overcomes the limitation of 
	blocking system calls, but introduce a huge overhead since many kernel threads can 
	bog down system performance.

	\paragraph{Many to Many}
	Multiplexing any number of user threads onto an equal or smaller number of kernel threads,
	relaxing restrictions on thread numbers on the user side while still allowing for true parallelism
	to occurr across multiple cores and solving the problem of blocking Syscalls stalling the whole process.

	\paragraph{Two-Level}
	Mixes the Many-to-Many model with the One-to-One model, increasing the flexibility of 
	the scheduling policies that can be implemented.

		
\chapter{Synchronization --- Lecture 9}
	\section{Process/Thread Synchronization}
	In the last lectures we introduced the concept of thread cooperation as a mean 
	to boost execution performance, however, we also said that cooperation must be 
	carefully achieved in order to avoid certain anomalies.

	This is especially true whenever we execute code located in locations that we 
	define as \emph{Critical Sections}, mainly code spaces where we access 
	shared data structures.

		\subsection{Synchronization}
		We need to define a set of methods to allow proper multiprogramming cooperation 
		in a controlled fashion whenever the program's flow of execution enters a 
		critical section in each of the possible units of execution (threads, processes)
		

			\paragraph{Switching to Sequential Execution}
			One of the most trivial approaches to solving these kinds of problems is allowing 
			only one unit of execution to execute the critical section at once, this basically 
			turns the flow of execution from a parallel to a sequential flow, where one of the 
			threads performs computation while the others wait.

			\paragraph{Locks}
			\label{par:lock}
			This Behaviour can be enabled through the use of \emph{Locks}, locks are a kind of 
			object that every thread must "hold" whenever he wants to execute the related
			critical section, thus ensuring a more predictable behaviour when computing a 
			critical section of code.
		

			\subsubsection{Synchronization Goals}
			Any possible solution to the critical section problem must abide by these 3 constraints:
			\label{par:mutex}
			\begin{enumerate}
				\item Mutual exclusion: Only one thread at a time can be in its critical section 
				\item Liveness: If no process is in its critical section and one or more want to 
				they must be able to do so
				\item Bounded waiting: A process requesting entry into its critical section will 
				eventually get a turn in a fair manner.
			\end{enumerate}

			For an example, in a shared data structure such as a stack, Mutual Exclusion means that 
			we do not get anomalous pop/push use that can disrupt its data, Liveness means that 
			if the stack is free to use it must be accessible by any thread that requires its 
			resources and lastly Bounded Waiting means that all of those processes must do so in 
			a queue that ensures that they will be able to access it in a finite, reasonable amount
			of time.

		\subsection{Synchronization Implementation}
		\label{ssec:prim}
		To properly implement the three previously stated goals we need some important 
		primitives:
		\begin{itemize}
			\item Locks, that we discussed in \hyperref[par:lock]{this} paragraph,
			\item Semaphores,
			\item Monitors.
		\end{itemize}
		
			\paragraph{Semaphores}
			Semaphores are a generalization of locks, they are built upon their concept but 
			provide a wider array of functionalities to allow for more complex synchronized 
			behaviours, such as defining a specific order of execution.

			\paragraph{Monitors}
			Monitors are a specific kind of object that is used to connect shared data to
			synchronization primitives.

		In addition to these three primitive types we also require some built in HW support for 
		multithreading, interrupt disabling and thread waiting, specifically:


		\paragraph{Interrupt Disabling}
		One of the main causes of inconsistencies (if not \emph{the} main cause) is 
		context switching, as we know the main enablers of context switching are interrupts
		which can either come internally (I/O System Calls) or externally (Time Slice 
		Interrupts), therefore we can prevent context switching from happening, at least 
		on single core architectures, by:
		\begin{enumerate}
			\item Enforcing Threads to not request any I/O in critical sections,
			\item Disabling interrupts in critical Sections.
		\end{enumerate}

		\paragraph{Interrupt Disabling Issues}
		There are a number of disadvantages and pitfalls coming from interrupt disabling, 
		apart from being the number one source of Halt \& Catch Fire instructions, it requires 
		a consistent overhead due to involving the 
		\hyperref[par:kernel]{kernel} and it is downright impossible 
		in a multicore architecture 

		\paragraph{Atomic Instructions}
		Atomic Instructions are a solution for performing operations that we do not want
		to be left in an inconsistent status by the CPU scheduler, one such example are
		\emph{read-modify-write} instructions, that, as they advertise, can read a value 
		from memory, perform computation on it and writing the result in one shot, without
		risking inconsistencies.\\ 

		On a uniprocessor architecture this is as straightforward as implementing the 
		instruction itself, on a multiprocessor one must also invalidate any copies of the 
		value that are present in other cores's cache.

		\paragraph{Atomic Instructions Issues}
		Atomic instruction also possess a set of problems associated with them, although
		they are less unsafe that interrupt disabling, they require other threads to 
		perform busy waiting while waiting for the resource to free up and, due to the 
		lack of a queue, they do not guarantee fairness among threads using the resource.

		\subsection{A More in-Depth View of Locks}
		Locks can be thought of as object possessing at least two methods:
		\begin{enumerate}
			\item Acquire: Allowing a thread to acquire the lock for that critical section
			\item Release: Allowing the thread to release the lock notifying other 
			waiting threads that the critical section is available
		\end{enumerate}
		
		this is enough to write synchronized programs that are concise and symmetrical

			\paragraph{Thread Symmetry}
			Threads are said to be symmetric if, for a single task, the set of all threads 
			performing computation on that task possesses the same original code.
		
\chapter{Advanced Synchronization --- Lecture 10}
	\section{Advanced Synchronization Structures}
	\label{sec:sync_struct}
	In the previous lecture we explored the functionalities, the advantages and the 
	disadvantages of various primitives used for synchronization, we now move on 
	to more complex data structures that provide wider functionalities, these 
	data structures are:

	\begin{itemize}
		\item Mutex
		\item Semaphore
		\item Monitor
	\end{itemize}

	\subsection{Semaphores}
	Semaphores are another type of data structure that, along with locks, can provide 
	mutual exclusion to critical sections, while also having the possibility of acting 
	as an atomic counter, it supports two operations:
	\begin{itemize}
		\item wait(): decrement counter, block until semaphore is open 
		\item signal(): increment, allow another thread to enter
	\end{itemize}

	\paragraph{Blocking in Semaphores}
	Each semaphore has its own queue of waiting processes/threads, when one of the 
	threads calls the wait() method, it continues if the queue is free, 
	otherwise the threads blocks on the queue.\\ 

	When the thread has finished and calls the signal() method, the resource is freed 
	and the semaphore gets its value incremented


	\subsubsection{Types of Semaphores}
	There are different variations of semaphores:
	\begin{itemize}
		\item Binary Semaphores: Also Known As Mutex, 
		They have the same behaviour as locks 
		\item Counting Semaphore: Manages multiple shared resources, the 
		initial value being the number of initial resources to be shared 
	\end{itemize}

	\paragraph{Problems of Semaphores}
	Semaphores provide greater flexibility in synchronized implementations, but come 
	at the cost of a high complexity and low transparency, in fact, understanding 
	waiting and signaling is not very straightforward and this therefore leaves 
	the quality of the implementation to the mercy of the programmer.

	\subsection{Monitors}
	\begin{definition}[Monitors]
		Monitors are a programming language construct that control access to shared data
	\end{definition}
	Monitors are similar to a Java (or C++) Class, embodying data, operations and synchronization
	all in one package, adding synchronization at compile time.

	Monitors, unlike classes, guarantee mutual exclusion for all their methods and require 
	all data to be private\footnote{This is to allow data to be updated only by getters 
	and setters, which in turn are mutually excluded.}

	\paragraph{Monitors: Formally}
	\begin{definition}[Monitors (Formal)]
		Monitors are Classes that define a lock and a set of conditions that 
		are used to manage concurrent access on shared data, the monitor utilizes 
		the aforementioned lock to allow for \hyperref[par:mutex]{Mutual Exclusion} of multiple threads.
	\end{definition}

	Thanks to the additional layer of abstraction provided by the use of Monitors, 
	we can provide a standardized implementation to streamline some more complex multithreaded programs, 
	allowing for an higher level of consistency in the quality of programs at the varying skill level of 
	different engineers that could be working on the same project, in this way, Monitors are a (partial) 
	solution to the problem of complexity of a synchronized multi threaded program.

	Of course this, as with many things in Computer Science, comes with a trade off: when inserting 
	an additional layer of abstraction in our technology stack, we surely decrease the performance of
	our implementation due to increasing the overhead of operations.

	\subsubsection{Mesa vs Hoare}
	Mesa and Hoare are two ways in which monitors can be implemented:

	\paragraph{Mesa-style}
	Mesa-style monitors are the implementation that is actually preferred 
	by most modern programming languages, in this implementation, 
	if a thread signals, a waiting thread is placed on the ready queue.

	The signaler, however, continues inside the monitor's critical section,
	meaning that the condition that allowed the new, waiting thread to take 
	control might not necessarily be true when he enters the critical section,
	and must therefore be re-checked.
	
	\paragraph{Hoare Style}
	Hoare-style monitors are the implementation preferred by most textoooks, 
	since it is the one in which implementations are the most intuitive.

	When a thread signals, it immediately goes to the waiting queue, meaning that 
	the one that takes his place has the certainty that the condition that 
	he was anticipating holds when he executes.

	\paragraph{Code Examples}

	In code, the way they would differ is:

	\begin{lstlisting}
	/* Mesa Style */

	//We repeatedly check that the condition
	//we want to be verified is upheld.
					 
	while(condition)
	{ 
		wait();		
	}

	/* Hoare Style */

	//We only check once for the condition and 
	//Then we wait, since if we finish waiting
	//We are guaranteed that the condition still holds.
	if(condition)	
	{				
		wait();		
	}
	\end{lstlisting}
	


\chapter{Deadlock --- Lecture 11}

\chapterquote{When two trains approach each other at a crossing, 
both shall come to a full stop and neither 
shall start up again until the other has gone.}{Kansas Legislation}{ Early 20th Century}

	\section{Deadlock}

	In the last lecture we explored a number of \hyperref[ssec:prim]{primitives} utilized to implement synchronization in
	Multi-threaded programs and a number of more abstract \hyperref[sec:sync_struct]{Data Structures}, we also 
	approached a number of synchronization problems that we used to justify the introduction of a number of these constructs,
	these are:

	\begin{enumerate}
		\item Producer-Consumers: when two threads both share a single "buffer" data structure, one inserting elements in the buffer 
		and the other removing from it.
		\item Readers-Writers: when two threads both share a single "buffer" data structure, one reading input from the buffer
		and the other writing new data in.
	\end{enumerate}

		\subsection{An unlikely Analogy: Philosopher's Lunch}
		Suppose that, for example, we are in the lunch hall at the department of philosophy, and we see the following scenario:

		There are 5 philosopher sitting on a round table, each having a chopstick to their side, by definition of a philosopher can only
		ever do two things:
		\begin{enumerate}
			\item Think
			\item Eat
		\end{enumerate}

		To eat one philosopher must first obtain both chopsticks on his sides; once he has eaten, he goes back to thinking.

		We wish to provide an algorithm that allows all philosophers to share the chopsticks fairly, Maximizing the frequency of 
		eating, while allowing for maximum concurrency.

		\paragraph{Trivial Solution}
		The easiest way out would be to simply provide a global lock necessary for eating and rotating it around the 
		table allowing each philosopher to eat, one after each other.

		This is trivially feasible, but for sure has a very low degree of parallelization, therefore, we must look for a 
		better solution.

		\paragraph{Waiting on Chopsticks}
		A simple, parallel solution would be the following code (written in pseudo-C++):

		\vspace{1cm}

		\begin{lstlisting}
#include "PhilosopherThread.h"
#include "Chopstick.h"

class PhilosopherThread
{
private:
	Chopstick chopstick1;
	Chopstick chopstick2;
	
public:
	void PhilosopherThread::Think()
	{
		\\Method Definition
	}

	void PhilosopherThread::Eat()
	{
		\\Method Definition
	}

	PhilosopherThread::PhilosopherThread()
	{
		while(True)
		{
			chopstick1.wait();
			chopstick2.wait();
	
			Eat();
	
			chopstick1.signal();
			chopstick2.signal();
	
			Think();
		}
	}
}
		\end{lstlisting}

		\vspace{1cm}

		this is, however, a flawed implementation: context switching can happen at any time, and if it were to happen in such a way 
		that all philosophers have grabbed just a single chopstick, no one would be able to grab the necessary resources to complete the 
		job and free the resources he already occupied, this means that we enter a state of \emph{deadlock}

		\label{sec:deadlock}
		\begin{definition}[Deadlock]
			Deadlock is a state that a set of threads can enter, in which every thread is holding a lock on a 
			resource while waiting for another one to free up, effectively blocking each other in a recursive manner.
		\end{definition}

		\paragraph{Deadlock and Starvation}
		While they are both a runtime state that brings a number of problems to the program in execution, deadlock and starvation are not to be confused:
		A program in a state of deadlock is in an irrecoverable condition, whilst a program in a state of starvation is only suffering from lack of 
		system resources, but it could still be normally executed if a sensible \hyperref[sec:scheduling]{scheduling} policy were to be implemented

		\paragraph{Monitors}
		A more complex solution can be provided through the use of monitors, while we do not provide the full code, the specifications are as follows:\\

		We define a philosopher as a thread possessing three states, eating and thinking whenever it's busy and hungry to signal that it's ready 
		to eat, if a philosopher is hungry and its neighbors are not eating, we can enter a synchronized block in which we pick up 
		\emph{both} chopstick and we switch to the eating state, otherwise, if our neighbors are eating we simply wait for them to finish without
		obstructing access to any resource, preventing a deadlock scenario.

			\subsection{Real World Examples}
			The set of abstract Problems that we just explored is useful not only because it allows us to explore a variety of solutions to 
			problems of synchronization between threads, but also because many real world problem can be seen as a generalization of them:
			
			\begin{enumerate}
				\item Producer-Consumer: Many Audio/Video Players embedded in a 
				web browser: a shared data buffer, network and rendering threads
				\item Reader-Writer: Banking System: read vs update account balances
				\item Philosophers's Lunch: Multiple resources locks: forms a basis 
				for any complex multithreaded application
			\end{enumerate}
		
		\subsection{Deadlock Conditions}

			For any given multithreaded program, we can define four conditions\footnote{Rather than seeing them as
			a set of four independent conditions that can be used to identify deadlock, their are best viewed as a 
			sequence of mistakes in design that allow a deadlock situation to emerge.} which \emph{must}
			verify in order for the program to consider itself in a state of deadlock 
			\label{ssec:deadlock_conditions}
			\begin{enumerate}
				\item Mutual Exclusion: At least one thread must hold a non-sharable resource
				\item Hold and Wait: At least one thread is holding a non-sharable resource and is waiting 
				for other shared resources to be freed by a second thread.
				\item No Preemption: A thread can only release a resource voluntarily; neither 
				another thread nor the OS can force it to release the resource 
				\item Circular Wait: A set of waiting threads $T = \sequence{t}{n}$ where $t_i$ is waiting on $t_{(i+1)\%n}$
			\end{enumerate}

		\subsection{Deadlock Solutions}

			After exploring this set of general solutions to multithreaded programs, we have encountered the problem of \hyperref[sec:deadlock]{deadlock},
			Deadlock is an especially dangerous situation since it permanently traps the program's execution and effectively results in the program 
			becoming stuck forever, it can even lead to a 
			\href{https://en.wikipedia.org/wiki/Halt_and_Catch_Fire_(computing)}{Halt and Catch Fire} situation (if it involves a vital portion
			of the system, such as kernel threads responsible for the operation of sensible areas of the operating system).

			While deadlock is not inevitable since it can be introduced in any multithreaded software as a result of bad programming practices,
			we still wish to provide for a set of solutions to at least safeguard the user and the machine from such a danger.

			Such possible mechanisms are:
			\begin{itemize}
				\item Deadlock Detection: Building up a system capable of spotting instances of deadlock in multithreaded programs 
				in order to recover their execution back to a manageable state.
				\item Deadlock Prevention (offline): Imposing a number of restrictions in the design of multithreaded programs 
				in order to prevent less skilled engineers from using unsafe practices that could lead to deadlock. 
				\item Deadlock Avoidance (online): Implementing a Runtime system that checks a number of conditions on 
				running threads (such as resource utilization requests) in order to prevent a deadlock state.
			\end{itemize}

			\subsubsection{ Resource Allocation Graph}
			In order to provide a system for detecting possible deadlocks, 
			we can use a data structure that is fairly common in Computer Science: a Graph.

			\begin{align*}
				&Let \; G(V, E) \; \text{Be a directed graph where:}\\
				&V := \text{The set of vertices of both: } resources = \sequence{r}{m}, \; threads = \sequence{t}{n}.\\
				&E := \text{The set of edges, a subset of } V \times V.\\
				&\text{Edges can be of two types:}.\\
				&\text{Request Edge: A directed edge } (t_i, r_j) \text{ indicates that } t_i \text{ has requested } r_j \text{ but not yet acquired}.\\
				&\text{Assignment Edge: A directed edge } (r_i, t_j) \text{ indicates that the OS has allocated } r_j \text{ to } t_i.
			\end{align*}

			If the resulting graph appears to be cyclic, then the represented thread structure is in a deadlock state, and we can then employ some solutions to 
			solve this problem, which we are going to explore in the next lecture.

\chapter{Deadlock Solutions --- Lecture 12}
\section{Solving Deadlock}
\subsection{Deadlock Detection}
Last lecture we saw how we can represent any multithreaded program with a 
specific instance of a directed graph, known as a Resource Allocation Graph

We also saw how we can infer the presence of a deadlock from the structure of this 
Resource Allocation Graph itself, by checking for cycles between the threads.\\

Specifically, Deadlock detection works like this:

\begin{enumerate}
	\item Scan the Resource Allocation Graph for cycles.
	\item Break them in several ways:
	\begin{enumerate}
		\item Kill all the threads in the cycle, (Overkill)
		\item Kill all the threads one at a time until deadlock stops,
		\item Pre-empt all resources one at a time rolling back to a 
		consistent status.
	\end{enumerate}
\end{enumerate}

This whole solution is quite costly due to the nature of graph scanning algorithms, 
for example, a common implementation of depth-first search (DFS) runs in a worst-case possible 
of $\mathcal{O}(|V|^2)$

This then begs the question of when to scan the R.A.G in order to find deadlocks, with risks of 
implementing significant overheads on resource allocation or resource request, in fact, this 
solution is not pursued at all by modern OSs, leaving the burden of a solution on the programmer itself.

\subsection{Deadlock Prevention}
Another solution that we mentioned last lecture, but that we did not explore in any depth, is that 
of preventing deadlock altogether, by preventing at least one of the necessary \hyperref[ssec:deadlock_conditions]{four conditions}
from coming true\footnote{Once again it is better to see these four points as a list of steps that one can take 
to prevent deadlock from happening before it's too late, than a series of concurrent conditions that must verify.}:

\begin{enumerate}
	\item Mutual Exclusion: Make all resources sharable,
	\item Hold and Wait: Make it impossible for a thread to hold any resources when it's also requesting another,
	\item No Pre-emption: Make it so that if a thread fails getting a resource that it requests, it must also 
	release all the resources that it is already holding,
	\item Circular Wait: Impose an ordering/priorities on resources and enforce their request in such order.
\end{enumerate}

\subsection{Resource Reservation}
In this solution to deadlock, each threads keeps track of a set of information 
related to the maximum number of resources it might need during execution:

\begin{align*}
	&m_1 = \text{Maximum number of resources that thread i might request}\\
	&C = \sum\limits_{i=1}^nc_i = \text{Total number of resources currently allocated}\\
	&R = \text{Maximum number of resources overall available}
\end{align*}

After we have access to this information, we can check if any thread sequence is safe 
by checking for the inequality:

\[ m_i - c_i \leq R - C + \sum\limits_{j=1}^{i-1}C_j\]

Or simply if for any thread, the resources he might still request are less than 
the currently available ones plus the resources currently allocated to all threads 
before him.

Just now we informally introduced the notion of safe state, which can simply be defined as: 

\begin{definition}[Safe State]

	An execution state of a multithreaded program for which the threads are not at a risk of Deadlock.
	
\end{definition}

This allows us to predict at resource allocation if the new state would be a safe state, and 
deciding in response what our policy should be, thus ensuring the impossibility of a circular wait 
\footnote{Note that that an unsafe state does not indicate deadlock, but merely the risk of its presence.}

\subsection{Enhanced Resource Allocation Graphs}
	
	\begin{definition}[Enhanced Resource Allocation Graph:]
		We can extend the original definition of Resource Allocation Graph as follows:
		\begin{align*}
			&\text{Request Edge: A directed edge } (t_i, r_j) \text{ indicates that } t_i \text{ has requested } r_j \text{ but not yet acquired}.\\
			&\text{Claim Edge: A directed edge } (t_i, r_j) \text{ indicates that } t_i \text{ might request } r_j \text{ in the future }.\\
			&\text{Assignment Edge: A directed edge } (r_i, t_j) \text{ indicates that the OS has allocated } r_j \text{ to } t_i.
		\end{align*}

		A cycle in this graph then indicates an unsafe state, so in the same way as we did before, we can detect the 
		presence of this possible unsafe state and only allow resource allocations if and only if they transition 
		the Enhanced Resource Allocation Graph from a safe state to another.
		
	\end{definition}

			solve this problem.
		
\chapter{Main Memory --- Lecture 13}
	\section{Memory Management}
			So far in our classes, we have seen how the Operating System manages \hyperref[ssec:proc]{processes} and \hyperref[sec:threads]{threads}, how it schedules them 
			and different sets of problems and respective solutions that arise with these kinds of architectures, we now 
			wish to go more in depth and examine an even more low-level aspect of Operating Systems:

			\begin{center}
				\emph{Memory Management}
			\end{center}

			In all of our past problems, we fundamentally discussed multiple ways to share a single resource: the CPU and its 
	
			computational power, now we will instead focus on ways to share space on the Memory instead.

		\subsection{Memory Management Goals}
			We wish to provide the following functionalities by designing a proper memory manager:
			\begin{itemize}
				\item Allocate memory resources among competing processes in a way that maximizes memory utilization and system throughput,
				\item Guarantee process isolation, for Virtual Addresses and Memory Safety,
				\item Provide a convenient abstraction to the programmer, keeping up an illusion of an unlimited amount of memory.\footnote{This is similar to the illusion of a truly parallel system that we wish to uphold when we design a process \hyperref[sec:scheduling]{scheduling}.}
			\end{itemize}
		
		\subsection{From Source Code to Executables}

		A typical, high-level programming language refers to instructions and data with symbolic names, called \emph{Tokens},
		these are usually idioms in a humanly understandable language such as English that intuitively represent their behaviour.\footnote{E.g: the function
		\emph{count()} counts members of a list in python.}

		\marginnote{Translation of source code is a responsibility of either a Compiler or an Interpreter, depending on the nature of the language,
		these then go through an Assembler and a Linker, resulting into machine language}

		For a user program to be executed, it needs to:
		\begin{enumerate}
			\item Be translated from source code to a binary executable and stored on a mass storage device\footnote{An Hard Disk or a Solid State Drive.}.
			\item Brought from the mass storage device into main memory
		\end{enumerate}

		The second step, which is the one we care about in this section of the course, is done by the \emph{Loader}, which is usually provided by the 
		\hyperref[sec:OS]{Operating System}.

		\subsection{CPU Addressing}

		Any modern CPU executes instructions in a predictable pattern, such as:
		\begin{enumerate}
			\item Fetch 
			\item Decode 
			\item Execute
		\end{enumerate}

		This means that all instructions must be retrieved from main memory and they might also require further accesses in case 
		they are instructions which purpose is the manipulation of data in the main memory itself.\\


		Of course Hardware (and therefore also machine language) is only aware of actual physical memory addresses, but in most 
		modern OSs, memory is \emph{Virtualized}, which means that many programs might refer to the same addresses, which are then 
		mapped to different addresses on program load by the Operating System. 

		\subsection{From Logical to Physical Addresses}
		As we have just discussed, there exists a sort of pipeline that address references in a program must follow in order 
		for these addresses to be translated from tokens in that language's syntax to physical addresses in main memory,
		firstly, we identify three states that addresses might be in:

		\begin{enumerate}
			\item Symbolic Name: The Instructions are written by the programmer using the language's token
			\item Logical Addresses: are generated by the user programs via the CPU
			\item Physical Addresses: the addresses are referring to actual memory addresses on the memory chip
		\end{enumerate}
		
		\paragraph{Address Binding}
		As we have discussed before, turning symbolic names into logical addresses is done through compilation, while the 
		mapping from Logical to Physical Addresses is called \emph{Address binding}.
		
		Address binding can be performed at different stages in the Compilation/Execution process of a program, for example:
		\begin{itemize}
			\item At Compile time.
			\item At Load time.
			\item At Runtime.
		\end{itemize}

		\subsection{Different Styles of Address Binding}

		\paragraph{Compile Time (Static Binding)}
		if the starting physical location where a program will reside is known at compile time, the compiler can 
		generate a so-called \emph{Absolute Code}\footnote{This was the approach in the very early days of computing,
		it is wholeheartedly obsolete for a set of obvious reasons.}

		This means that Logical Addresses match Physical Addresses, which in turn means that if the program is loaded 
		at another location in memory, it must be recompiled in order to offset all the addresses to the new location.

		\paragraph{Loading Time (Statically relocatable Binding)}
		
		If the starting location where the program will reside is \emph{not} known at compile time, a simple solution to 
		allow different starting positions for our program would be to provide for an environment variable to be 
		specified by the Operating System, which represent the offset from the original starting position, which is then 
		summed to all the references in our code, allowing us to have valid addresses

		in such an architecture, we have the following correspondence between physical and logical addresses: 

		\[ addr_{phys} = addr_{log} + \text{offset}\]

		\paragraph{Execution Time (Dynamic Binding)}

		If the program can be moved around in main memory during its execution, we need a more complex solution, the compiler 
		generates some \emph{Dynamically Relocatable Code} or Virtual Addresses, which are then mapped to physical memories using 
		a dedicated hardware module, a Memory Management Unit (MMU).

		In this case, the relation between physical and logical addresses is not describable by any mathematical formula since 
		every mapping is determined arbitrarily by the MMU, which we can therefore envision as a "black box" that allows us 
		to perform these translations.
	
	\section{Different Kinds of Memory Management}
		\subsection{Uni-programming Memory Management}

		We first examine the easiest case of memory management: Operating Systems in which only one program might execute at a time.

		\paragraph{Static Binding Implementation}
		In such a case, we have the following organization: the Operating System is given a fixed part of physical memory to perform 
		its essential functions,\footnote{This is usually located in the highest memory addresses, for example this is the case in MS-DOS.}
		Processes run one at a time and are loaded each in a contiguous segment of memory, this allows compile time address binding 
		since it is a very deterministic and simplistic architecture.

		\paragraph{Static Binding Problems}
		The previous solution, as is the case with any Static Binding solution, presents the following problems:
		\begin{enumerate}
			\item No OS Safety: Every process can reference OS Memory,
			\item No Inter-Process Safety: Processes can reference (and therefore interfere with) each other,
			\item Low Scalability: Most of the features that are present in modern OSs are simply impossible to 
			implement with such a solution.
		\end{enumerate}

		\paragraph{Loading Time Binding}

		We move on to a different solution: we keep most of the previous assumptions such as the OS having a reserved memory space,
		the rest of the memory being allocated as user space etc \dots

		The difference is that we allow the OS to perform Load time binding of addresses on user programs, as we can see this solution 
		shares most of the Advantages/Disadvantages with the previous one:
		\begin{itemize}
			\item Pro:
			\begin{itemize}
				\item No Hardware support needed.
			\end{itemize}
			\item Con(s):
			\begin{itemize}
				\item No Protection,
				\item Contiguous Address space,
				\item No Runtime process relocation by the OS.
			\end{itemize}
		\end{itemize}

		\paragraph{What Our Solution Needs}
		It is then obvious that we need a memory management architecture that solves the problems that we previously stated, 
		and that therefore allows for some measure of security and relocation to be implemented, for this we need hardware 
		support in the form of the (previously mentioned) MMU and some changes to our OS's architecture: \\

		Our MMU must contain two registers:
		\begin{enumerate}
			\item Base Register: containing the address of the start of the program's address space in physical memory,
			\item Limit: containing the size limit of the address space.
		\end{enumerate}

		While our CPU must support two operating modes:
		\begin{enumerate}
			\item Privileged (kernel) mode, which is active when the OS is running,
			\item User Mode, which is active when a user process is running.
		\end{enumerate}

		With this solution we can efficiently determine if any addressing instruction is out of the bounds of our user space, but we still 
		assume that our process's address space is contiguous, assumption that we will continue to uphold for quite some time, since otherwise 
		things get way more complicated.

		\paragraph{Implementing Memory Relocation}
		Whenever a program references an address, the CPU must check if it is within the allowed bound for that process:
		\[
			addr \in [base, \; base + limit)
		\]

		if this does not hold true, we get an error known as \emph{Segfault}, or Segmentation Fault.\\

		Advantages of our current implementation are:
		\begin{itemize}
			\item We finally provide protection between OS and processes and between processes,
			\item We can easily relocate processes or grow their size by altering the values of the base and limit registers,
			\item All of this can be implemented easily with a simple hardware circuit (the MMU).
		\end{itemize}
		While we pay for the following disadvantages:
		\begin{itemize}
			\item We pay a little overhead for Hardware translation at each memory reference,
			\item We still assume contiguous allocation ,
			\item Our processes are still limited to the size of physical memory size,
			\item The degree of our multiprogramming is limited since all of the active processes must fit in main memory, 
			\item User programs cannot share \emph{any} portion of memory, not even in the case of common libraries. 
		\end{itemize}

	\section{Lecture Summary}
	From what we have seen, we can gather that memory management is crucial for system performance, since fetching instructions from 
	memory is something that happens \emph{every time} something is executed by the OS, also we can see that very simple memory management 
	architectures require very little OS or Hardware support, but provide no security guarantees and very poor performance from a 
	multiprogramming or User Experience point of view, and therefore that more complex solutions are more desirable in a modern environment.

\chapter{Fragmentation --- Lecture 14}

In the previous lecture, all of the memory management solutions that we 
explored were based on a single, powerful assumption: that memory was 
contiguous, we now wish to remove this assumption and develop a system that 
allows for processes to be stored in different parts of memory.

\section{Contiguous Memory Allocation}

First we examine a situation in which processes themselves are not 
contiguous in memory, but every process possesses a contiguous 
address space in which he works.

In this situation, when a process is de-allocated and his memory 
is freed, it creates a "hole", or an available slot in main memory,
subsequently, these slots can be re-used by new processes 

\subsection{Memory Allocation Processes}

\paragraph{First-Fit}
	The simplest algorithm that can allocate memory that possesses 
	holes in between processes is a linear one: we scan through memory 
	in a top-down (or bottom-up) fashion until we find a hole that is 
	big enough to fit our process

\paragraph{Best-Fit}
	Another algorithm would be to pursue an optimal fit for each allocation, 
	looking to allocate the smallest possible hole capable of satisfying
	the current request.

	This minimizes the amount of size of the remaining free memory
	supposedly reducing fragmentation
	\footnote{%
		We say supposedly because after some time, one ends up with 
		many tiny unallocated holes that we cannot possibly allocate
		a process to, actually leading to higher fragmentation.
	}

\paragraph{Worst-Fit}
	A Counter-Intuitive approach would be to actually pursue the most 
	sub-optimal fit for each allocation, in this way we allocate the largest
	possible hole, this way we leave the largest possible hole behind
	so that it might be still possible to use it to serve another 
	process request in the future.

\subsection{External Fragmentation}
	The frequent loading and unloading of processes will inevitably cause an 
	increasing degree of fragmentation in main memory, experimentally, we can say
	that for every 2n blocks that we allocate, we lose n blocks due to fragmentation.

	This phenomenon is called: \emph{External Fragmentation}

\subsection{Internal Fragmentation}
	Internal Fragmentation is provoked when we allocate a process that is very close 
	to the maximum size of a hole, leading us to leave a minuscule amount of memory 
	behind, in such a way that the process of keeping track of that tiny amount of memory 
	is more expensive than the benefit of having it free. (due to it being so small
	that it cannot be possibly allocated)

\subsection{Solving Fragmentation}

	\paragraph{Full Defragmentation}
	One way to solve External Fragmentation and prevent Internal Fragmentation is 
	to run an algorithm that relocates all processes to a single contiguous block 
	of memory, allowing us to compact all the fragmented parts of memory and to 
	re-use them again.
	
	\paragraph{Partial Defragmentation}
	A less expensive approach would be to just relocate the smallest amount of processes  
	in a way that allows the incoming processes to be scheduled, reducing the amount 
	of work that must be done before we can reap the benefits of free memory.

\section{Swapping}
	We take a step back from talking about main memory and instead talk about a 
	solution to another problem that we presented last lecture: that of the illusion 
	of infinite space that must be provided to the user.\\

	The solution makes use of Memory Hierarchy in order to present a memory that is 
	orders of magnitude bigger than main memory to the user, we accomplish this by 
	swapping:
	
	\paragraph{Swapping Implementation}
	
	Swapping happens whenever we switch a process from main memory to a mass storage 
	device in the machine, we might want to swap a process out if it is not being 
	used for a significant amount of time, allowing us to free memory to be used 
	for more necessary processes. 

	\paragraph{Swapping Performance}
	Swapping is a very costly operation, since it requires access to a hard drive\footnote{
		Even if the mass storage device we would be talking about was an SSD, swapping would be 
		faster but still orders of magnitude slower than simply accessing RAM.
	}
	and therefore must not be performed too leniently, in fact this poses a significant 
	enough disadvantage to make swapping obsolete in modern OSs.

\chapter{Virtual Memory (1 of 2) --- Lecture 15}
\section{Paging}

A solution to all the problems presented with the previous 
architectures is \emph{Paging}, a memory management scheme 
in which the address space of a process is contiguous but 
it is split into equally sized blocks of an arbitrary size 
called \emph{pages}.

\paragraph{Physical and Virtual Memory}
In a system that performs paging, the memory of a process 
is virtualized, meaning that each process is allocated a 
slice of physical memory which might be composed of several 
non-contiguous slices, the process sees these as a single contiguous 
block of virtual memory thanks to \emph{page mapping}


\paragraph{Paging Implementation}
Page/Frame numbers and their sizer are architecture-dependent, of course 
they are practically always chosen between powers of two, in the range
$[512B, 8192B]$, the reason for this is that virtual and physical addresses 
can be accessed easily through the use of bit masking.\\ 

Memory Addresses are of course represented through binary numbers,
this means that if the Page/Frame is a power of two, it will be 
represented in a fixed amount of bits in the full address\footnote{Say for example, in a
32 bit address, the first 10 bits are the page number and the next 22 are the offset.}, this 
makes it possible to separate the page from the offset by the aforementioned 
bit masking method and map its respective frame.

\paragraph{Page Mapping}
Page mapping is a function performed by the Operating System, 
it maps each Logical Address of a process to a Physical Address
in main memory, this is done through the use of a page table. 

\subsection{Page Table}
The page table is a crucial part of an architecture that utilizes 
paging, since every memory access must pay a fixed overhead in order 
to go through the page table to obtain the actual physical address with 
which to address memory.\\

For this reason the Page Table is implemented in an hardware component, which 
also implements a TLB\footnote{Transition Look-Aside Buffer.}
a form of caching that speeds up page loading by caching recently used pages, 
avoiding access to main memory.\footnote{Whenever a process requests a page that is not currently cached in the TLB, we have 
a page fault.}

\paragraph{TLB and Multiprogramming}
The TLB is shared across all processes since, being a hardware piece, it is 
impossible to replicate it in a flexible way.

The same page number can be mapped to a different frame number depending on which 
process requests the translation, this interaction with multiple processes can be 
approached in two ways:

\begin{enumerate}
	\item Basic: at every context switch the TLB is flushed and cleaned, this is very 
	trivial to implement but leads to a `cold start' state on every context switch, since 
	the TLB being emptied leads to a high number of misses until it's populated again
	\item Advanced: at every context switch the TBL contents are cached in the PCB and restored 
	on the next context switch, alternatively, a Process Context ID (PCID) is added to each entry,
	allowing the CPU to recognize which entry is associated to which process.
\end{enumerate}

\subsubsection{The Page Table as a Security Measure}
The page table can also be used to protect processes from accessing 
parts of memory they shouldn't access, or to incapsulate their memory away 
from other processes.\\ 

A number of bits can be added to the page table to be used as classification bits, 
specifying if a frame is to be used in read/write-only mode or if it's available for both.
Every memory can then be checked against those bits to ensure that the reference is handling 
that memory as intended, this bits can also double as flags to signal which entries are being used 
by the current process.

\paragraph{Shortcomings}
This method, while being relatively simple to implement, is not enough to provide 100\% protection from 
illegal memory accesses, due to internal fragmentation, leading to the need of additional 
security measures (that we will explore in the following lectures).

\subsection{Memory Initialization}
When a process is initialized, so must be its memory, this procedure is entrusted to 
the Operating System, happening as a sequence of steps:

\begin{enumerate}
	\item The process requests for $k$ memory pages
	\item if $k$ frames are free the OS allocates them to the process, otherwise try to free up memory 
	by swapping out unused frames
	\item The OS loads the pages in their corresponding frames and proceeds to populate the 
	page table with their respective mappings
	\item The OS either flushes the TLB or restores the TLB from a previous PCB (depending on which of the 
	two approaches that we discussed earlier the current OS is using)
	\item The OS handles misses during the process's runtime by populating the TLB according to 
	the principles of temporal and spatial locality.
\end{enumerate}

The fifth step is an operation that happens continuously as the process runs and is 
interrupted at times by context switches (in a modern, multiprogramming system)

\subsection{Page Sharing}
A system that employs paging can make it very easy to share a block of memory, since it 
removes the necessity of contiguity, sharing can be accomplished simply by duplicating a page 
entry, having multiple processes's pages point to the same frame, this can be done 
both for code and data.\footnote{This is a common occurrence in modern systems, such as in 
Windows's DLL (Dynamically Linked Libraries).}\\

This can only be accomplished if the code is \emph{reentrant}, meaning that:
\begin{itemize}
	\item It does not write or change itself (it is non self-modifying)
	\item The code can be shared by multiple processes at the same time, as long as each 
	has it's own copy of the data and registers, including (of course), the instruction register.
\end{itemize}

\subsection{Paging: Summary}
All in all, paging is a huge improvement over memory relocation:
\begin{itemize}
	\item It removes the problem of external fragmentation,
	\item It allows code sharing, reducing redundancy in main memory,
	\item It enables processes to run without being completely loaded.
\end{itemize}

It has it's set of drawbacks, namely:

\begin{itemize}
	\item It introduces a fixed overhead on memory references (Virtual to Physical address translation),
	\item It might need special hardware support to make it efficient (TLB),
	\item It inevitably increases the level of complexity of the Operating System.
\end{itemize}






\chapter{Virtual Memory (2 of 2) --- Lecture 16 to 18}

\section{Segmentation}
To most programmers, the notion of memory not existing 
in one, continuous linear address space is well known, Programs
are instead divided into multiple \emph{segments}, each dedicated 
to a specific use, for example
\begin{itemize}
	\item Code (or Text),
	\item Data,
	\item Stack,
	\item Heap.
\end{itemize}

Memory Segmentation supports this view by providing addresses with 
a segment number (mapped to a segment base address) and an 
offset from the beginning of that segment

\paragraph{Segment Table}
A segment table works akin to page table, it is used to map segment-offset addresses 
to physical addresses, it is also used to check for invalid addresses, using 
a system similar to that used by the page tables to check for validity.\\

The segment table is made up of a relatively small number of base/limit hardware registers 
that are used to set the bounds of the various segments of a process, this allows the segment 
table to be stored with physical registers, improving it's performance.

\paragraph{Implementing Segmentation}
To implement Segmentation we need a compiler that can generate addresses 
whose Most Significant Bits (MSBs) can indicate the segment number, we can then 
combine segmentation with static or dynamic relocation to broaden the possibilities of 
our system, additionally, we might need additional hardware like a TLB if our programs 
use many logical segments.

In practice, however, most modern systems utilize Segmentation in tandem with Paging 
in order to get the best of both worlds.

\section{Paging and Segmentation}
So far we discussed about two ways to divide physical memory:

\begin{enumerate}
	\item Paging: How the OS divides and maps physical memory to logical memory
	\item Segmentation: How a compiler divides a process into logical segments (Data, Code, Stack, Heap, etc \dots)
\end{enumerate}

A modern Operating System combines both of those to get all of the advantages.

\section{Virtual Memory}
\label{sec:vmem}
In a practical situation, processes do not need all of their pages loaded in main memory, since 
a real program probably spans hundreds if not thousands of pages while actually, due to 
the principle of locality, it only accesses roughly 10\% of them 90\% of the time, therefore 
there is no need to load the \emph{entire} process in main memory when it starts, the way this 
is implemented is through \emph{Virtual Memory}

\paragraph{Virtual Memory Implementation}
Virtual Memory uses a backing storage (some form of mass storage such as a HDD or an SSD) to store 
unused pages, exploiting memory hierarchy to give the illusion of infinite virtual address space.\\

With an OS that uses Virtual Memory, we get: 

\begin{itemize}
	\item The ability to load portions of processes On-Demand, which leads to:
	\begin{itemize}
		\item Programs that can be written with an address space that is orders of magnitude larger 
		than what is physically available on the computer
		\item Processes that only allocate as much memory as it's strictly necessary, increasing the 
		quantity of total processes that we can load
	\end{itemize}
\end{itemize}

What we're proposing is fundamentally not a novel idea, we are just deciding to have another 
cache, this time we use Main Memory itself as a cache for our mass storage device, we do it as follows:

\begin{itemize}
	\item Our page table gets a validity bit that indicates if the page is in disk or in memory,
	\item When the page is loaded to memory, the validity bit is asserted.
\end{itemize}

This means that our TLB is not valid just because it contains the mapping to main memory, but 
also if the corresponding page is also present in main memory, otherwise we would still be calling 
a "hit" a situation in which we pay a very high overhead!\\

For this system to make sense, however, we must have a high probability that our memory accesses are 
referencing addresses that are located in main memory

\paragraph{Page Faults and Page Tables}
At each logical memory reference, we perform a page table lookup, which follows these steps

\begin{enumerate}
	\item We check for the table entry's validity bit, we then do one of these two things:
	\begin{itemize}
		\item If the bit is asserted, the page is in memory,
		\item If the bit is negated, we fetch the page from mass storage.
	\end{itemize}
	\item We fetch the memory block from the page in main memory.
\end{enumerate}

\paragraph{Page Fault Handling}
The way the OS handles page faults depends on the architecture:
\begin{itemize}
	\item x86 Hardware saves the virtual address in the CR2 register
	\item Other platforms get only the address of the faulting instruction 
	and must simulate it in order to find the address that generated the fault (Text or Data)
\end{itemize}
			 
\paragraph{Idempotent vs Non-Idempotent instructions}

Instructions can be divided between Idempotent and Non-Idempotent:
\begin{itemize}
	\item An Idempotent instruction can be re-run if it's faulty and we can expect it to 
	behave in a very similar way, so we can recover the faulty address from it with ease
	\item A Non-Idempotent instruction cannot be simply re-run, since the side effects of 
	re-running the instruction might not point to the same address or even
	jeopardize the correctness of the program
\end{itemize}

Issues of Non-Idempotency might be even more extreme, such as instructions that move a block 
of memory, blocks spanning multiple pages or page changes while page faults are happening, 
all of this side effects are extremely complicated to recover from.

\paragraph{Virtual Memory Performance and Consistency}

Theoretically, since every instruction must be fetched from memory, in a system that virtualizes 
memory a page fault may occur at each instruction therefore, for the system to be worthwhile, we 
keep this formula into account:

\[
	t_{access} = (1-p)*cost_{access}+p*cost_{fault}
\]

which can manifest itself in something like:

\[
	t_{access} = (1-p)*100+p*\num{20000000}
\]

to find a satisfiable upper bound, we can then solve an equation in terms of p, for example, 
if we want the speed to be at most 10\% slower than what we would have without virtualization, 
we need:

\begin{align*}
	1.1 * 100 &= (1-p)*100+p*\num{20000000}\\ 
	p &\approx 0.0000005 = 5 * 10^{-7}
\end{align*}

\subsection{Virtual Memory Management}
So far we have discussed how the OS manages page faults in virtual memory, however, 
we still need to take into account a number of functionalities that we need in order for 
virtual memory to be manageable:

\begin{itemize}
	\item Page Fetching: when the OS should load a process's pages into main memory,
	\item Page Replacement: what page the OS should remove from memory if it is filled.
\end{itemize}

Let us examine those one by one.
\subsubsection{Page Fetching}
The goal of page fetching is to exploit virtual memory to provide an illusion of 
infinite physical memory, we do this by exploiting the \emph{locality} of memory references 
in programs:\\

We only keep in memory the pages that are being used, while keeping in mass storage 
those that we are not currently using, ideally providing a system that has the speed of 
the main memory system while having the capacity of the mass storage system.

Page Fetching can be implemented through three strategies:

\begin{enumerate}
	\item Startup: All the pages of a process are loaded at once, this does not allow 
	us to provide any illusion, as the virtual address space cannot be larger than physical memory 
	\item Overlays: We let the programmer handle page loading/removal, this allows us to 
	provide the illusion of infinite memory but, however, places the burden of correctness on the 
	programmer's shoulders, making the whole process hard and error-prone,
	\item Demand: The process communicates with the OS, telling it when it needs a page, page management
	is then left to the Operating System.
\end{enumerate}

Intuitively, we can reason that most modern architectures prefer the \emph{Demand} approach, preferring to 
leave the responsibility of page managing to the Operating System.

\paragraph{Pure Demand Paging}
If we are using the Demand Paging, when we initialize a process none 
of it's pages are loaded, rather, we swap a page in memory only when 
the process references it (and therefore we have a page fault), such a
swapper is called a \emph{Lazy Swapper}, or pager.

\paragraph{Prefetching}
We can possibly improve the performance of Demand Paging by performing
some heuristics predictions on when a program is going to access a new page,
loading it before the fault happens and therefore avoiding a page fault, this 
can only be done in a reliable fashion if a computer accesses memory \emph{Sequentially},
or in any other predictable pattern.

\paragraph{Swap Space}
In modern systems, we reserve a portion of the disk for storing pages that 
we "evict" from main memory, this portion might not be part of the actual disk file system,
for example:

\begin{itemize}
	\item On Linux, we have a dedicated swap partition on disk,
	\item On Mac, the swap space is part of the file system.
\end{itemize}

\paragraph{Swap Out}
When a page needs to be swapped out, it will be copied to disk (to exploit 
memory hierarchy), a process's pages are divided into two groups:
\begin{enumerate}
	\item Code (Read-Only)
	\item Data (Initialized/Uninitialized)
\end{enumerate}

Depending on the type of page that we remove, we can implement different optimizations:
\begin{enumerate}
	\item We remove a Code Page:
	We know that the code's content does not change, since it is read only, therefore, 
	we just remove it and reload it back from its executable, without the need for storing the 
	page in a separate file.
	\item We remove a Data Page:
	We know that data might be modified at runtime, therefore, we cannot rely on it being the 
	same as the one stored in our executable: we \emph{must} store it into swap space 
	in order to fetch it later on.
\end{enumerate}

\subsubsection{Page Replacement}
Whenever we have a page fault, we need to load a page from disk to memory, if 
the physical memory has free frames available, there is no problem, we can just load 
the page into a frame, if, however, our physical memory is full, we must swap out a frame
to make room for the new page \footnote{The process of swapping a frame out of memory 
is often called frame \emph{eviction}.}.

\paragraph{Replacement Algorithms}
There are many different algorithms to implement frame evictions, such as:

\begin{itemize}
	\item Random: swapping a random page out (works surprisingly well!),
	\item FIFO (First In First Out): Removes the page that has been in memory the longest
	\item MIN (OPT): removes the page that will not be accessed for the longest time, 
	it is provably optimal, but relies on predicting the future,
	\item LRU (Least Recently Used): Approximates MIN by removing the page that has not 
	been used in the longest time (using the past to predict the future!).
\end{itemize}

A summary of their use-cases:
\begin{itemize}
	\item FIFO --- Easy to implement but lead to too many page faults,
	\item MIN --- Optimal but practically impossible to implement,
	\item LRU --- Good approximation of MIN, implements the principle of 
	Temporal locality, works poorly when locality doesn't hold.
\end{itemize}

\paragraph{LRU Implementation}
We do not need a perfect LRU implementation, only a good approximation in order for 
the algorithm to provide near-optimal results, we can implement this through:

\begin{enumerate}
	\item Single-reference bit: we assert the bit if the page is accessed, this can 
	distinguished pages that have been accessed since the last TLB flush
	\item Additional-reference bits: we maintain eight bits for each entry, right 
	shifting them at periodic intervals, the high order bit is filled in with 
	the current value for the reference bit, the bits are then cleared.

	\begin{center}
		At any given time, the page with the smallest value is the LRU page
	\end{center}
	
	The number of bits and the frequency of update can be adjusted in different implementations.
\end{enumerate}

\paragraph{Second Chance Algorithm}
Another approach is to mix the Single-Reference bit implementation of LRU with a standard 
FIFO queue:

We keep frames in a FIFO circular list, we set the reference bit to one at every access, when a page 
fault happens:

\begin{itemize}
	\item if the bit is 0: we replace the page and set it to 1,
	\item if it's 1: set it to 0 (give it a second chance) and move to the next frame.
\end{itemize}

\subsection{Multiprogramming and Thrashing}
In all of our recent considerations, we have always taken for granted that 
our system is always running a single process at a time, however, with modern multicore 
architectures, we need to consider the fact that multiple processes might run concurrently on the same 
CPU, however, we know that the set of pages that are being worked on is very small\footnote{Precisely it follows the 
90/10 rule.}, this allows the system to load very few pages, increasing the degree of multiprogramming.\\

\paragraph{Thrashing}
Whenever the degree of multiprogramming is too high, too many active working sets get 
loaded into main memory, saturating it's capacity, this state is called \emph{Thrashing}.\\ 

In a state of thrashing pages from different processes are continuously loaded while the 
TLB is full, leading to a constant swapping of pages that are, in practice, still needed
for the execution of other processes.

This binds execution speed to the speed at which the disk can be accessed to 
retrieve these lost pages, translating in a catastrophic degradation of performance.

\paragraph{Thrashing Avoidance}
Thrashing is so catastrophic that we want to design some policies in order to 
prevent it from happening, two of these policies are:

\begin{enumerate}
	\item Global Allocation/Replacement: All the pages are in a single LRU queue, 
	upon page replacement, any page might be a potential "victim" of eviction, this is:
	\begin{itemize}
		\item Flexible (and easy to implement),
		\item Risky (thrashing is more likely and more disastrous).
	\end{itemize}
	\item Local Allocation/Replacement: Each process has its own LRU queue, therefore, 
	a process might only evict his own pages, reducing the effects of thrashing, this is:
	\begin{itemize}
		\item Safer (reduces the damage that thrashing can do to performance),
		\item Less Efficient (runs the risk of not allocating enough memory to a process).
	\end{itemize}
\end{enumerate}


\chapter{Mass Memory (1 of 2) - Lecture 19}
We now enter part five of this course, focusing on how we can store information 
on peripherals and system devices, by looking at what is called the \emph{File System}.
\section{Mass Storage Devices}
the File System is incapsulated by an interface known as the 
\begin{center}
	File System API
\end{center}
the File System API allows the user to perform actions such as file 
creation and manipulation, acting as a "frontend".

These operations go through the middleware that is the OS's implementation, representing 
the files through data structures, this acts as a "Middleware" 

All of these data structures are stored in the Physical Implementation, consisting of 
the Mass Storage devices and the scheduling algorithms that manage their access.

\subsection{Mass Storage Devices Categories}

Mass storage devices can be divided into three categories:

\begin{enumerate}
	\item Magnetic Disks (HDDs)
	\item Solid State Disks (SSDs)
	\item Magnetic Tapes\footnote{Common in old mainframes and legacy systems.}
\end{enumerate}

\subsubsection{HDDs}

In this current lecture we will focus more on Magnetic Disks, even if currently they are 
being replaced by SSDs, they give us the opportunity to discuss common disk scheduling algorithms.

\paragraph{HDDs Anatomy}
An Hard Disk is made of one or more platters covered with some sorts of magnetic material, the 
consistency of the material changes the name of the Drive:

\marginnote{Larger sector sizes reduce the space wasted by headers and trailers, while increasing internal fragmentation.}

\begin{itemize}
	\item Rigid Metal $\to$ Hard Disk
	\item Flexible Plastic $\to$ Floppy Disk
\end{itemize}

Each of the platters has two working sides, each of those surfaces is further divided into an 
arbitrary number of concentric rings, called \emph{tracks}, all the tracks that are at the 
same distance from the edge of the platter (or the center of the platter) is called a \emph{cylinder}, 
each of the tracks is further divided into \emph{sectors}, usually containing 512 bytes.


Sectors also contain some metadata in their header and trailer, along with a checksum for validity checks.

\paragraph{Heads}
Data on hard drives is read by read-write \emph{heads}, a standard configuration utilizes one head per surface, 
each of those heads controlled by a separate \emph{arm}, all of which controlled by a common \emph{arm assembly}, moving 
simultaneously from one cylinder to the other.

\paragraph{Storage Capacity}

We can calculate (A really rough approximation of) Disk Capacity through the following Formula:

\begin{align*}
	H &= \text{Number of Heads}\\
	T &= \text{Number of Tracks} \\
	S &= \text{Number of sectors} \\
	B &= \text{ Number of Bytes} \\
	Capacity &= H*T*S*B
\end{align*}

This is not necessarily true due to the circular nature of the disk: inner cylinders contain less sectors than 
outer ones.

In the 1980s every track had the same number of sectors with the same number of bits, therefore the bit density in 
the inner sectors was much higher than in the outer sectors, while disk controllers had no "intelligence" of their own,
this had drawbacks (of course):

\begin{itemize}
	\item The capacity of the disk was determined by the maximum density that a controller could handle,
	\item Different frequencies and timings need to be considered for innermost vs outermost tracks.
\end{itemize}

\paragraph{Constant Density}
If we want to calculate the practical capacity of a disk, we must consider that the number of sectors
$S$ varies with the radius of the track on the platter, with outer platters accommodating more sectors that
inner ones, if we take this fact into account, we can then proceed to design a disk that possesses the same 
bit density all along its area.

This is what came along after the previous design, while also adding smarter controllers that allowed for 
logical addressing of sectors rather than physical, through \emph{Zone Bit Recording} or ZBR

\paragraph{Logical Referencing}
A physical block of data is specified by the triple:
\[
	(head, cylinder, sector)
\]

blocks are usually numbered by starting from the outermost cylinder, which is zero-indexed \footnote{Do note that cylinder number coincides with track number.}

\paragraph{Data Transfer}
An Hard disk rotates at a constant angular speed\footnote{For example $7200rpm/120rps$.}, due to the nature of angular speed in physics, you have that 
outer tracks spin faster than inner tracks, leading to more sectors being accessed in the same amount of time due to larger tracks.

Data transfer from disk to memory is a 3-step process:
\begin{itemize}
	\item Positioning Time (also called seek time),
	\item Rotational Delay,
	\item Transfer Time.
\end{itemize}

We now analyze each step one by one:

\begin{itemize}
	\item Positioning (seek) Time:
	The time required to move heads to a specific cylinder, including the time for the head to settle after the move,
	It depends on how fast the hardware moves the assembly arm and, since it depends on physical speed rather than electronical components, it's usually the 
	slowest step in the process.
	\item Rotational Delay:
	The time required for the desired sector to rotate and come under the read-write head \footnote{Note that an HDD is constantly spinning, so depending on which 
	sector we land on with respect to the one we are searching for, our delay can be anything between zero or a full revolution, averaging out at half a revolution.}\
	\item Transfer Time:
	The time required to move data electronically from disk to memory, sometimes expressed as transfer rate (or bandwidth), in bytes per second
\end{itemize}
The final delay necessary for the fetching of data is:

\[
	Delay_{seek} + Delay_{rotation} + Delay_{transfer} = Delay_{total}
	\]

\paragraph{Disk Structure}



\paragraph{HDD Risks}

When an HDD is operating, disk heads `fly' over the surface on a very thin
cushion of air, if the head accidentally contacts the disk a \emph{Head Crash}
occurs, in order to prevent this destructive accidents, we take a precaution:

Whenever the computer is turned off, disk heads are `parked', meaning that they
are moved to a safe area that is off the disk where no data is stored.

\paragraph{Disk Interfaces} %I am beginning to think that there are too many paragraphs and too few sections, tough luck!
Hard Drives might be removable as floppy disks, some are even hot swappable\footnote{They can be removed while the computer is running.}

Disks are connected to the computer via the I/O bus, usually through some common interface such as:
\begin{multicols}{3}
	\begin{itemize}
		\item ATA and SATA
		\item USB
		\item SCSI
	\end{itemize}
\end{multicols}


The transfer of data between disk drives and main memory is controlled by a controller, %duh
or in this case, controllers:

\begin{itemize}
	\item The host controller is at the computer end of the I/O Bus
	\item The disk controller is built into the disk itself
\end{itemize}

the CPU issues commands to the host controller through (typically) a memory-mapped I/O port.

In addition to these devices, data is usually transferred from the magnetic surface to an 
onboard cache by the disk controller and, subsequently, to main memory to the host controller
and the motherboard at electronic speeds.

\paragraph{Data Transfer Time Minimization}
There are two main bottlenecks to data transfer time in magnetic disks, these are:

\begin{enumerate}
	\item Seek Time
	\item Rotational Delay
\end{enumerate}

In order to minimize transfer time to the disk we need to minimize those, but how?

\begin{itemize}
	\item Smaller Disks: smaller diameters means smaller travel distance for arms
	\item Fast-Spinning Disks: faster spinning means lower rotational delay.
\end{itemize}

We can also take a look at the number of OS optimizations that we can implement
software-side in order to reduce delays, for example, scheduling disk operations 
in a way such that we minimize head movement, laying data that is closely related in 
a contiguous fashion in order to further reduce head movement and place commonly-used 
data in a well-known portion of the disk.

\subsection{Magnetic Disks: Summary}
Disks are slow devices compared to CPUs and main memory, due to them being
bounded to a physical process in order to function, therefore, the efficient 
management of these devices is often crucial.

A good set of OS optimizations can drastically reduce the overhead on disk access
and, since disk access is the main bottleneck in a computer, increase the performance
of the system as a whole.

\chapter{Mass Memory (2 of 2) --- Lecture 20}
\section{Disk Scheduling}
In our previous lectures we have seen how magnetic disks are ubiquitous 
even in modern systems, due to their optimal cost/capacity ratio.

We have also seen how, however, magnetic disks are also the main bottleneck 
in a computer due to the time their mechanical components need 
to position themselves over the right sector on the disk, there
isn't much we can do on the hardware side as Computer Scientists\footnote{
	Hardware optimizations are also still bounded by the physical speed of the HDD's arm,
	therefore seeking another approach for optimization is also necessary for matters of 
	efficiency.
}, what we can do instead is provide a set of scheduling algorithms that the Operating System
can implement on the software side in order to optimize the way that the HDD's arm accesses 
the disk's sectors.

\paragraph{The Current Situation}
Whenever we have a process that issues a read/write operation to a file, this operation gets 
mediated by the Operating System, this is, first of all, because a user process cannot 
access kernel structures directly and must therefore pass through the OS's functions and 
interrupts in order to do that, this includes operations such as the aformentioned read/write 
operations that require access to disk.

In a multiprogramming environment the OS can run multiple processes at any given time and,
therefore, can also receive multiple I/O requests to the same controller, these accesses get 
enqueued, we can \emph{sort} the order of this queue according to some arbitrary rule that 
would minimize the latency of our accesses.\footnote{We do this with the aim of 
minimizing seek time, not rotational delay, that is, we wish to position our head on 
the right track/cylinder, letting the right sector rotate itself into position, we therefore 
assume rotational delay to be \emph{negligible} with respect to seek time.}

\begin{center}
	We can make use of \hyperref[sec:scheduling_alg]{Scheduling Algorithms} in order to 
	optimize disk access.
\end{center}

\subsection{Disk Scheduling Algorithms}
\subsubsection{First Come First Serve (FCFS)}

First Come First Serve (FCFS) works in the most naive way possible: I/O requests get 
processed in the order that they come in, this algorithm has the following characteristics:

\begin{itemize}
	\item It's easy to implement,
	\item Works well when the system's load is low,
	\item It quickly leads to performance deterioration as requests increase,
	\item It's mainly used by SSD drives since they \emph{do not} require any mechanical ,
	movement, in this way, their operation is like Random Access Memory (RAM).
\end{itemize}

\subsubsection{Shortest Seek Time First (SSTF)}

Shortest Seek Time First (SSTF) is also pretty self explanatory: it selects the I/O request 
that would lead to the minimal amount of seek time necessary, it has the following 
characteristic:

\begin{itemize}
	\item It's implemented through a sorted list of requests that gets updated every 
	time a new request is added,
	\item The overhead of sorting the list every time a new request is made is negligible 
	when compared to any mechanical operation (such as seek time),
	\item It might cause starvation due to a high number of short seek time requests,
	preventing the disk from accessing a far away cylinder,
	\item It is \emph{not} optimal as it minimizes seek time greedily.
\end{itemize}

\subsubsection{Elevator (SCAN)}

SCAN or Elevator Algorithm allows the head to freely move back and forth across the disk, 
serving requests as the head passes, it has the following characteristics: 

\begin{itemize}
	\item It also requires a sorted list in terms of distances from the edge of the platter,
	\item It wastes time on travelling from the last request to be served when moving in one 
	direction to the edge of the platter and back.
\end{itemize}


\paragraph{SCAN vs SSTF}

SCAN is much more fair that SSTF is, since we guarantee that eventually the arm will pass 
on every request, it is, however, still not optimal, since in some cases seek time 
might be larger with SCAN rather than with SSTF, for example when A new request comes in immediately after SCAN has passed:


In this case SSTF will immediately serve the new request, whilst SCAN will complete the whole pass before 
serving the new request, resulting in longer wait time for requests just visited by the disk's arm.

\subsubsection{An Optimization of Elevator (LOOK)}

An trivial optimization of SCAN is LOOK, which prevents the head from advancing further once 
the last request that could be served when going in that direction has been served, sparing the 
HDD's arm from travelling all the way to the edge and back.

\subsubsection{Circular SCAN (C-SCAN)}
In Circular SCAN the Head only moves towards one direction, when it gets towards the edge of 
the platter it gets "restarted" back to the beginning and then moves all the way down again. 

This is due to some mechanical constraints requiring the limiting of acceleration and deceleration 
of the disk's head when performing read \& write operations: by moving in a constant fashion we can 
avoid start\&stop movements of the disk's head, preventing mechanical breakdowns.

\subsubsection{Circular LOOK (C-LOOK)}
Circular SCAN can be optimized in the same way as regular SCAN, by avoiding to move the 
disk head all the way and resetting it's position on the request that's closest to the start 
instead of all the way back we can save a considerable amount of time, this algorithm is 
therefore (rather unimaginatively) called Circular LOOK (C-LOOK)

\subsection{Scheduling Algorithms Implementation}
In general, Disk Scheduling Algorithms are typically implemented on the disk's controller,
which is also responsible for interacting with the OS's kernel through a designated driver.

Usually Disk drives are shipped with one of these algorithms ready, more complex scheduling 
algorithms can be designed in order to optimize different metrics, however, one should be careful 
to move complex logic to the Disk Driver in the OS Kernel instead.

\subsection{Interleaving} 
One of the other problems that presents itself when disk access is bound by physical constraints is  
that of rotational speed exceeding the rate at which we can read a full block and issue the instructions to 
read the next one:\\

\marginnote{Today's CPUs are so fast that interleaving is not even considered anymore!}

We can skip some arbitrary number of blocks when we allocate memory to accomodate for rotational speed, 
\emph{Interleaving} different files together in order to allow the disk's head to access them at a 
more comfortable pace.

\subsection{Read-Ahead}
Another optimization that we can perform when fetching a block of memory is exploting the principle 
of \emph{Locality}, we assume that if we fetch a block from memory then also the blocks which are close 
will be needed soon, we can therefore store them in the controller's buffer in order to allow for a much 
faster access time \emph{if} they get loaded.

\section{Disk Formatting}
Before a disk can be used, it has to be \emph{low-level formatted}, the process of formatting 
implies the laying down of all headers and trailers marking the beginning and ends of 
each sector, these contain relevant information such as:

\begin{itemize}
	\item Sector Numbers 
	\item Error Correcting Codes
\end{itemize}

Error correcting is checked on every read \& write, if damage is detected and recoverable, the 
disk controller handles the error, this is known as a \emph{soft error}.

\paragraph{Partitioning}
Once a disk is low-level formatted, the next step before usage is \emph{Partitioning} the drive 
into one or more separate partitions, this must be done even if the disk is to be used 
with a single, large partition, since the OS must write the partition table at the beginning 
of the disk anyways.\\

After partitioning a disk's filesystems must be \emph{logically formatted}, we are going to 
explain this concept in depth later on. 

\subsection{Boot Block}
Every computer ROM contains a \emph{bootstrap} program, this program 
contains just enough code to find the first sector of the first hard 
drive in order to load it in memory and transfer control to it, this is 
called \emph{bootstrapping the system}.

\paragraph{The Master Boot Record (MBR)}
The Master Boot Record is the first sector in a disk, it contains 
a very small amount of code and the \emph{partition table} which 
tells the the disk how it is logically partition, it also indicates 
which partition is the active (or bootable) one.

\paragraph{Bootable Partition}
The Bootable partition\footnote{Usually \code{C:\textbackslash} in a windows system.} 
is the one that contains the Operating System files such as the 
Kernel and \code{.DLLs} that are responsible for the OS's functioning,

When this partition is identified during the bootstrapping process, it's 
loaded in main memory in order to transfer control to the Operating System, that 
loads all the important Kernel Data Structures and Services in order to correctly 
operate the device.

\section{Solid State Drives (SSDs)}
SSDs are a `recent' \footnote{In the grand scheme of things, they have only been popular 
since the last decade, but I am sure someone reading this in the future will have a nice laugh 
at how I am describing the 2050s's equivalent of a floppy disk as a recent technology.}
develompent in the field of mass storage, they are a smaller but faster solution for 
storing relatively huge amounts of data, they are usually employed in tandem with a Hard 
Disk Drive, in order to store the Bootable Partition and therefore improve the performance 
of a system's Operating System.

\paragraph{SSDs and Laptops}
In some devices such as laptops, the HDD might even be completely replaced by a SSD, 
the benefits are clear: an SSD requires \emph{no} mechanical movement in order to operate, 
it is therefore:

\begin{itemize}
	\item Less Cumbersome.
	\item More Power Efficient.
\end{itemize}

This, in addition to the more obvious advantages such as speed, combined with the fact that 
laptops don't usually require as much space as a normal PC, makes SSDs the ideal mass storage 
for a laptop.

\paragraph{SSDs implementations}
SSDs make use of technologies such as:

\begin{itemize}
	\item Flash Memory (More common in phones).
	\item DRAM chips protected by a battery.
\end{itemize}

As we said before, there is \emph{no need} for mechanical operations when fetching data from 
memory, this is what makes SSDs so much faster, it also has a nice effect on the software side, 
\begin{center}
	There is \emph{no need} for disk scheduling algorithms.
\end{center}

this is because blocks are accesses \emph{directly} by referencing their block number, in this way, 
SSDs are very similar to RAM.

\subsection{SSDs: A Final Overview}
We can summarize Solid State Drives as follows:

\begin{itemize}
	\item Read operations are super fast (the whys have been explained before).
	\item Write operations are slower as they need an erase cycle (SSDs cannot overwrite data directly, blocks are 
	instead dereferenced and then a garbage collector takes care of them\footnote{Ew, Java.}).
	\item SSDs have a limited number of writes-per-blocks over their lifetimes, therefore, an SSD's controller 
	needs to count how many time a block gets overwritten, in order to spread these writes back.
	\item SSDs are best used as an additional step in the base of the memory hierarchy pyramid, to store 
	system files, the bootloader and act as a sort of cache for the hard disk .
\end{itemize}

\section{Magnetic Tapes}
Hearing the word "Magnetic Tape" immediately conjures images of Computer Scientists working 
inside the wiring of room-sized machines, surrounded by thermionic valves and beeping lights.

In some way Magnetic Tapes \emph{are} a thing of the past, they are slow, they have a high seek time 
and a capacity that is also dwarfed by Hard Disk Drives.

Sometimes, in a bank of some old-fashioned european country, one might still find a magnetic tape 
being used to store data, perhaps as a backup, but nothing more than that anymore.

\section{RAID Structures}
RAID stands for "Redundant Array of Inexpensive Disks", they are, as in the name, a collection 
of cheap hard drives that are grouped together in order to increase redundancy or speed up operations.

In the modern times, RAID systems employ large (and possibly expensive) disks, therefore, the 
definition has changed to "Redundant Array of Independent Disks".

\section{Disk Failures}
In Real-World distributed systems, often we require \emph{multiple} disks:

\begin{center}
The more disks we employ the more the reliability of the system increases, more specifically, 
we decrease the Mean Time To Failure (MTTF) of the system.
\end{center}

Let us suppose we have a system of $N$ disks, each with a MTTF of 10 years $\approx \SI[group-separator = \ ]{4000}{\day}$,
we can perform a more in-depth analysis:

We associate to each disk a Random Variable $\mathcal{X}_{i,t} \sim Bernoulli(p)$
\[
\begin{cases}
	&\mathbb{P}(\mathcal{X}_{i,t} = 1) = 0.00025 = p\\
	&\mathbb{P}(\mathcal{X}_{i,t} = 0) = 0.99975 = 1-p\\
\end{cases}
\]

Where 1 indicates a failure and 0 indicates no failure, also assuming for 
simplicity that $p$ is the same across all disks and that all $\mathcal{X}_{i,t}$
are independent between eachother.

We wish to compute the Expected Number of failures in a certain day $t$, given 
that the probability of failure of one disk is $p$, under our assumption we can 
say that:

\begin{align*}
	& \mathcal{T} = \mathcal{X}_{1,t} + \mathcal{X}_{2, t} + \cdots + \mathcal{X}_{N,t}\\
	& T \sim Binomial(N, p) \\
	& \mathbb{E}(T) = Np 
\end{align*}

Therefore if we can plot our expected failures we can better understand if 
HDD failure presents a true risk to an organization that 

\begin{enumerate}
	\item Values it's data,
	\item Possesses a great volume of Drives upon which to store it's data.
\end{enumerate}
\clearpage

\paragraph{Analysing Expected Failure}
If we would plot our expectation as a function of the number of disks at one's disposal, this
is what we would see:

\begin{figure}[ht]
\begin{center}
	
	\begin{tikzpicture}
		\begin{axis}[xscale = 1.0, yscale=1.00,
		  xmax = 6000,
		  xmin = 0,
		  ymax = 5, 
		  ymin = 0, 
		  grid = both,
		  axis lines = left,
		  xlabel = {\(Number\ of\ Disks\)},
		  ylabel = {\(Failures\)},
		  ]
		  \addplot[
			domain=0:6000,
			samples = 60,
			thick, red]{x*0.00025};\label{fails}
			\addlegendentry{\(\mathbb{E}(T) = Np \)}
		  \addplot[mark=*,color=blue] coordinates {(4000,1)};
			\vertLineFromPoint{4000,1};
			\horLineFromPoint{4000,1};
		\end{axis}
	\end{tikzpicture}

	\caption{We can se that the Expected Failures (\ref{fails}) increase linearly, for a significant 
	(but still not huge considering the results) amount of HDDs we get an average of a failure a day,
	anything near this being a completely unacceptable situation.}
\end{center}
	\label{fig:fails}
\end{figure}

\paragraph{Preventing Failures}
We wish to find solutions that can mitigate what we just discovered: organizations 
possessing a good number of drivers risk a non-negligible chance of failure of at 
least a single one of them that, if 
it would happen to the wrong drive, could spell disaster for the whole company.

\paragraph{Mirroring}
Such a method is Mirroring: 
providing some redundancy by copying the same data on multiple disks,
thereby preventing a single disk's failure from wiping data permanently 
from the company's records.

In this way the data can be read in parallel by multiple disks, also increasing 
the speed at which data itself can be read (we could also speed writes up through 
some sophisticated scheduling algorithms, but it would be pretty complicated).

\paragraph{Striping}
Striping is a RAID technique that aims to improve read performance, striping is 
implemented by spreading data out across multiple disks, so that, when it needs to 
be accessed, the largest number of disks possibile can be used to access it.

\subsection{RAID Levels}
So far we have two "opposite" technique that we can implement through a RAID structure, 
these are:

\begin{enumerate}
	\item Mirroring,
	\item Striping.
\end{enumerate}

Mirroring aims to solve the \hyperref[fig:fails]{problem} of disk failures, but does so 
by essentially wasting space, while Striping aims to improve performance, at the cost of 
data being less redundant. 

Depending on which kind of performance and safety a user needs, RAID has different implementations,
known as "levels"

\begin{enumerate}
	\item Raid 0: Striping without Mirroring,
	\item Raid 1: Mirroring without Striping,
	\item Raid 2: Mirroring with error-correcting codes,
	\item Raid 3: Mirroring with bit-interleaved parity,
	\item Raid 4: Mirroring with block-interleaved parity,
	\item Raid 5: Mirroring with block-interleaved distributed parity,
	\item Raid 6: P + Q Redundancy
\end{enumerate}

\section{Final Disks Summary}
In short, to close this section of our lectures, we can summarize mass storage as follows:

\begin{itemize}
	\item Disks are slow devices compared to CPU and 
	Main Memory\footnote{
		Once again, see \href{https://en.wikipedia.org/wiki/Von_Neumann_architecture\#Von_Neumann_bottleneck}{Von Neumann Bottleneck}.
		}
	\item Due to them providing a bottleneck to the whole system, efficient management of these 
	resources is crucial for any modern OS and Hardware Controllers.
	\item I/O requests can be re-arranged through disk scheduling algorithms in order to 
	reduce mechanical movements of magnetic drives.
	\item Through redundancy one can cope with disk failures in critical application domains.
\end{itemize}

\chapter{File System: Theory --- Lecture 21}
\section{File System API}
We now enter the last part of this course: we wish to take a step back from the low-level 
point of view that we provided by exploring the implementation of mass storage both from an 
Hardware (HDDs, SSDs and Magnetic Tapes) and Software (Scheduling Algorithms) side.

We now want to explain in a general way (when possible) how an Operating System can provide a set of 
functionalities and services in order for an user to work with files.

\paragraph{OS Abstractions}
In what we have seen so far, one of the key roles of the Operating System is providing a set 
of abstractions to allow the user to interface with the hardware in a simpler manner, examples are: 

\begin{itemize}
	\item \hyperref[ssec:proc]{Processes} and \hyperref[sec:threads]{Threads} as a way to better utilize the CPU's resources,
	\item \hyperref[sec:vmem]{Virtual Address Space} as a way to better utilize the Memory's resources,
	\item Files as an interface with which to manipulate data which can be stored on Disk\footnote{For simplicity, in this section we 
	are going to refer to any kind of mass storage as a `Disk'.}.
\end{itemize}

\paragraph{Device-Independent I/O}
\marginnote{
	For example, there is no need for the User to know if the device is an External Disk or an USB pen
}
Anotehr advantage of the abstractions that an Operating System provides to an user is their portability:
by providing a common API through which an User can interact with any I/O Device, the user can design programs 
that make use of external I/O Devices without having to take into account the nature of this I/O device 

\subsection{User Requirements on Data}
There are a number of qualities that a User might require from a storage system depending on 
what the purpose of the data that must be stored is, some of these can be:

\begin{itemize}
	\item Persistence: The capability of the system to keep data around inbetween jobs, reboots and/or crashes,
	\item Speed: The capability of the system to retrieve data in fast enough manner,
	\item Size: The capability of a system to store a large enough amount of data,
	\item Sharing/Protection The capability of a system to protect/incapsulate data between users and to 
	give the possibility of sharing this data when it is deemed appropriate enough.
	\item Ease of Use: The capability of a system to be accessible and intuitive enough in order for 
	an user to easily manipulate and locate the aforementioned data.
\end{itemize}

\paragraph{HW vs OS Capabilities (and Responsibilities)}
Keeping these qualities in mind, we can draft a list of which of these characteristics can be 
implemented on the Hardware side and which of these are best implemented on the Operating System (Software Side):

\begin{itemize}
	\item Hardware:
	\begin{itemize}
		\item Persistence: Due to the non-volatility of disks,
		\item Speed (kind of): Through a set of carefully picked scheduling algorithms and software 
		technologies we can fetch data in a satisfying amount of time, 
		\item Size: Size of disks increases every year and is now commonly measured in Terabytes\footnote{
			$\SI{1}{\tera\byte} = \SI{1}{\byte}\cdot10^{12}$.
		}
	\end{itemize}
	\item Operating System:
	\begin{itemize}
		\item Persistence: The OS provides some Redundancy mechanisms in order to improve Data redundancy,
		\item Sharing/Protection: An OS usually provides a set of user privileges that can be used to determine who 
		can performe which actions on a file.
		\item Ease of Use: named Files, Directories and Search Tools are all instruments that simplify the 
		User's job when manipulating and searching for files\footnote{When we talk about Ease of Use we do not 
		refer to any GUI since we take GUI for granted in any modern OS.}.
	\end{itemize}
\end{itemize}

\subsection{Files}
Before we proceed with our deep dive into the concepts of Files, let us first 
define clearly what a file \emph{is}:

\begin{definition}[File]
	A File is the abstraction used by the Operating System to refer to the logical 
	unit of data on a storage device.
\end{definition}

A file is therefore a named collection of a set of related information that is stored 
on secondary memory (A picture, a game, a PDF file with your notes on, etc \dots).

Generally this information can either contain\footnote{or more accurately, \emph{represent}, 
since at the end it's all 1's and 0's.} data (\code{.txt}, \code{.tex}) or a program (\code{.exe}, \code{.py})

\paragraph{File Mapping}
Files are mapped by the Operating System onto a Disk, since such devices are non-volatile and 
we need to uphold the file's \emph{persistency}.

\paragraph{File Attributes}
Different Operating Systems keep track of different file attributes, also called 
\emph{metadata}\footnote{Data about other Data!}, such as:

\begin{multicols}{2}
	\begin{itemize}
		\item File Name,
		\item File Type,
		\item Location on Hard Drive,
		\item File Size,
		\item Protection,
		\item Time and Date of Creation
	\end{itemize}
\end{multicols}

\paragraph{File Operations}
\label{par:fops1}
Operations are procedures that can be performed on a file in order to manipulate 
either its content (data) or its attributes (metadata),
they can include, as data-modifying operations:

\begin{multicols}{2}
	\begin{itemize}
		\item \code{create()},
		\item \code{open()},
		\item \code{read()},
		\item \code{write()},
		\item \code{seek()},
		\item \code{close()},
		\item \code{delete()}.
	\end{itemize}
\end{multicols}

And as metadata-modifying operations (specific to UNIX, for the sake of example):

\begin{itemize}
	\item chown/chmod (change owner/permissions),
	\item ln (create symbolic links),
	\item etc \dots
\end{itemize}

All of these operations are typically implemented through system calls and wrapped 
in a User Library.

\subsection{Operating Systems (Kernel) File Data Structures}

The operating systems has a set of data structures that are used to keep track 
of current files that are being used, these are:

\begin{enumerate}
	\item The \emph{Global Open File Table}:
	A table that is shared by any process with an open file, it possesses one entry 
	for each open file, it keeps track of how many processes have a file opened through 
	the usage of a counter, along with the file's attributes and pointers to their locations
	on disk
	\item The \emph{Local Per-Process File Table}
	In addition to the global table, each process possesses its own local table, for 
	each entry, it specifies:
	\begin{itemize}
		\item A Pointer to this file's entry in the global table,
		\item The process's current position in the file,
		\item The process's open mode:
		\begin{itemize}
			\item read (r)
			\item write (w)
			\item read and/or write (r/w)
		\end{itemize}
	\end{itemize}

\end{enumerate}
\subsection{File Operations: An in-depth view}
For each of the operations that we enumerated \hyperref[par:fops1]{previously}, we will 
now provide a short explanation of what their purpose is and how they are implemented.

\begin{enumerate}

	\item \code{create(filename)}:\\
	Creating a file implies:
	\begin{itemize}
		\item Allocating disk space while also checking for disk quotas and permissions,
		\item Creating a File Descriptor for the file including: 
		\begin{itemize}
			\item filename,
			\item file location on disk,
			\item other attributes.
		\end{itemize}
		\item Adding the file descriptor to the directory that contains the file,
		\item (Optionally) Specifying the file's type, which allows for better error detection and a whole 
		set of optimizations that can be performed for that specific file, with the disadvantage
		of a more complex Filesystem and Operating System.
	\end{itemize}
	
	\item \code{delete(filename)}:\\
	A file's deletion implies:
	\begin{itemize}
		\item Finding the directory that contains the file,
		\item Freeing the blocks currently being occupied by the file,
		\item Remove the file's descriptor from the directory\footnote{This behaviour is dependent 
		on \emph{Hard Links}, that we are going to explore later.}.
	\end{itemize}

	\item \code{open(filename, mode)}:\\
	The opening of a file:
	\begin{itemize}
		\item Returns the file's ID that the OS associates with that filename,
		\item Checks in the Global Open File Table if the file is already opened by 
		another process, if not it finds the file and copies it's descriptor in the table,
		\item Checks if the file is protected against the mode,
		\item Increments the open count in its table entry,
		\item Creates an entry in the Process File Table pointing to the Global Table's entry, 
		\item Initializes the file's pointer to the beginning of the file.

	\end{itemize}

	\item \code{close(fileID)}:\\
	Closing a file implies:
	\begin{itemize}
		\item Removing the entry from the process's file table ,
		\item Decrementing the Open Count of this file in the Global File Table,
		\item Checking if the open count is 0, if this is the case, the entry can 
		be safely removed.
	\end{itemize}

	\item \code{read(fileID)}:\\
	Reading a file implies:
	\begin{itemize}
		\item Using the index returned by the \code{open(filename, mode)},
		this also implies that in order to read a file, it must have been opened 
		beforehand.
	\end{itemize}

	\item \code{write(fileID)}:\\
	Writing to a file is very similar to \code{read(fileID)}, with the difference 
	that it copies \emph{from} the buffer \emph{to} the file.

	\item \code{seek(fileID, pos)}:\\
	Seeking just updates the file position to \code{pos}, with no need for 
	actual I/O operations.

	\item \code{mmap(fileID)}:\\
	Memory mapping (mmap) allows to map a part of the Virtual Address Space to a file,
	greatly simplifying file access by bypassing system calls and, in turn, also 
	reducing the overhead necessary due to copying information from/to the buffer at 
	each operation.
\end{enumerate}

\subsubsection{File Reading}
To explore further  file reading, we can denote the difference between 
the two possible ways of reading a file: 

\begin{enumerate}
	\item Random/Direct access,
	\item Sequential access.
\end{enumerate}

Random access is common in most widespread devices such as Disks or Main Memory, whereas 
Sequential access is mostly used in devices who do not support Random access   
(since sequential is way slower) such as Magnetic Tapes.

\paragraph{Random File Reading}
A typical user function that implements a file I/O syscall through random access looks like: 

\begin{center}
	\code{read(fileID, from, size, bufAddress)}
\end{center}

Where the OS reads \code{size} bytes from position \code{from} into \code{bufAddress}:

\begin{lstlisting}[language = C++]
	
	for(int i = from; i < from + size, ++i) 
	{
		bufAddress[i - from] = fileID[i];
	}

\end{lstlisting}

\paragraph{Sequential File Reading}
A typical user function that implements a file I/O syscall through sequential access looks like: 

\begin{center}
	\code{read(fileID, size, bufAddress)}
\end{center}

Where the OS reads \code{size} bytes from the file's current position (\code{fp}) into \code{bufAddress}, 
updating the file's position accordingly:


\begin{lstlisting}
	
	for(int i = 0; i <  size, ++i)
	{
		bufAddress[i] = fileID[fp + i];
	}

	fp += size;
	
\end{lstlisting}

\subsubsection{File Reading: Perspectives}

\paragraph{The Programmer's Perspective}
From a programmer's perspective, there are three 
ways in which file access can be done:

\begin{itemize}
	\item Sequential, where data is accessed in order, one byte/record at a time,
	\item Direct/Random, where data is accessed at a specific position in the file,
	\item Keyed/Indexed, where Data is accessed based on a key.
\end{itemize}

\subsection{Naming and Directories}
Any OS needs a method to quickly and intuitively fetch files that are located 
on disk, it usually uses a pair of identifiers: one for internal use and 
one for the user itself.

The Operating System prefers utilizing UIDs\footnote{Unique IDentifier.} when 
addressing specific files, but it also keeps the file's name in the file descriptor
in order to present a more human-friendly name to the user.

\subsubsection{Directories: Overview}
A Directory is an OS data structure which maps files names to descriptors, in 
order to allow the translation from human-friendly nomenclature to the file's 
internal ID.

\begin{center}
	Directories are part of the OS data structures that are stored on disk but 
	cached in OS kernel memory.
\end{center}

\paragraph{Directory Operations}
Directory operations that are commonly supported by the Operating System include:

\begin{multicols}{2}
	\begin{itemize}
		\item Search for a file,
		\item Create a file,
		\item Delete a file,
		\item List a Directory,
		\item Rename a file,
		\item Traverse the file system.
	\end{itemize}
\end{multicols}

\subsubsection{Directories: Single-Level Directory}
An Operating System that implements Single-Level Directories possesses 
a single namespace for the entire disk, leading to the requirement of 
uniqueness being enforced on every filename. 

A Single-Level Directory system reserves a special area of disk to 
hold the directory, which itself contains a pair 

\[(name, index)\]

This architecture was understandably used in older systems due to 
space constraints but it's no longer employed due to it's limitations. 

\subsubsection{Directories: Two-Level Directory}
An Operating System that implements Two-Level Directories gives 
each user of the system it's own namespace, leading to different users
being able to give two files the same name, as long as they live in separate 
user-spaces

\paragraph{Master File}
Usually a Master File Directory is used to keep trach of each user's directory,
while needing a separate directory for system executables and \code{.DLL}s.

\subsubsection{Directories: Multi-Level Directory}
The next logical step to extend an Operating System's directory structure is 
increasing the number of levels through the use of a Tree-like data structure, 
in this implementation\footnote{Which may be the one that the computer 
you are using to read this is using right now.} each user/process has a 
concept of \emph{current directory} from which all searches take place.

\paragraph{Pathnames}
Files may be accessed using either \emph{absolute} pathnames, starting from the 
\emph{root} of the tree\footnote{Usually the Drive in which the file is 
located, such as \code{C:\textbackslash} or \code{D:\textbackslash}.} or \emph{relative} pathnames, starting from 
the current directory.

Directories are stored the same as any other file in the system, except there 
is a bit that identifies them as a directory.

\subsubsection{Referential Naming: File Links}
Sharing files betweeen different users's directory trees might be 
complicated, for this reason, UNIX provides 2 types of links via the 
\code{ln} command:

\begin{enumerate}
	\item Hard Links: Multiple directory entries that refer to the same file,
	\item Symbolic Links: An alias to the linked files.
\end{enumerate}

\paragraph{Hard Links}
When multiple directories refer to the same file through hard links, an Operating 
System must keep track of the number of those links that are present across it's 
whole filesystem.

When one of those links is deleted it's removal does not affect others, that is to say,
it does not delete the file itself \emph{unless} it was the last link in the filesystem.

This is checked by the OS simply by checking if $\text{reference count} = 0$ upon 
link deletion, if the assertion is true then the file itself is also deleted, freeing 
up disk space.

\paragraph{Circular Links}
Allowing hard linking to directories might cause circular links, those prevent 
the Operating System from claiming back disk space.

To solve this problem we simply disallow the user from making a hard link to a directory.

\paragraph{Soft Links}
Soft links work by providing an alias (an alternative name) to a filename, if 
the original file is then deleted, the file is also inaccessible by the 
soft link, since it's existence does not increment the Operating System's 
reference counter

\subsection{File Protection}
The Operating System must allow users to:

\begin{enumerate}
	\item Control the sharing of their files,
	\item Control the access of their files by granting a set of permissions 
	over some file operations.
\end{enumerate}

The two main approaches in the industry are:

\begin{enumerate}
	\item Access lists and Groups (Windows NT),
	\item Access Control Bits (UNIX/Linux).
\end{enumerate}

\paragraph{Access Lists}
Windows keeps an access list for each file, with it's user name and 
it's access type (a set of permissions detaling which operations 
that user is allowed to perform on that given file).

\begin{itemize}
	\item PROs: This solution is highly flexible.
	\item CONs: Access Lists can become large and tedious to mantain.
\end{itemize}

\paragraph{Access Control Bits}
UNIX/Linux details three categories of users, which are, in order:

\begin{multicols}{3}
	\begin{itemize}
		\item Owner,
		\item Group,
		\item World.
	\end{itemize}
\end{multicols}

it also details three different access privileges, which are, in order:
\begin{multicols}{3}
	\begin{itemize}
		\item Read,
		\item Write,
		\item Execute.
	\end{itemize}
\end{multicols}

it then uses this abstraction to provide each file with a 
nine-tuple that details it's permissions:

\[
	(111101000)\ =\ \text{rwxr-x---}
\]

Where each subsequence of three bits repeats the three privileges in the 
order that we have just given (from left to right) for each of the three
user types (also from left to right).

\subsection{Summary}
In this lecture, we have seen that:

\begin{itemize}
	\item The File System Interface provides a convenient abstraction to 
	users/applications that need to interact with I/O devices.
	\item The OS is responsible to expose and implement this interface, hiding 
	any specific details to users/applications.
	\item Files are abstract data structures that the OS uses to manage 
	it's data on mass storage.
	\item Operations on files are exposed through device-independent APIs.
\end{itemize}

\chapter{File System: Implementation --- Lecture 22}
In this final lecture, we will concern us about how the Operating System 
actually lays data down on Disk

\paragraph{Recap: Disk Overheads}
When we talked about Hard Disks, there were different types of delays, these 
were:

\begin{itemize}
	\item Overhead: The time the CPU (Or the DMA Controller) takes to start 
	a disk operation,
	\item Latency: The time to initiate a disk transfer of $\SI{1}{\byte}$ of memory,
	which we can further divide into:
	\begin{itemize}
		\item Seek Time: The time to position the head over the correct cylinder,
		\item Rotational Time: The time for the correct sector to rotate under the head.
	\end{itemize}
	\item Bandwidth: The Rate of I/O transfer once the transfer itself is initiated.
\end{itemize}

With this in mind, let's now move to a higher level of abstraction and work our way down.

\section{File Organization on Disk}
From the Operating System's perspective, a Disk is just an array of blocks.

For simplicity, we are going to think of a block as a disk sector, while, in practice, 
a block might be a multiple of a sector, for example: $(1 \text{ block} = 4 \text{ sectors})$.

Given this, we want data transfer to happen this way:

\begin{enumerate}
	\item The Operating System Requests for a $(\text{fileID, block})$ pair, such as $(42, 73)$,
	\item The disk responds with the corresponding $(\text{head, cylinder, sector})$ triple.
\end{enumerate}

Moreover we want the following qualities to apply for our implementation:
\begin{itemize}
	\item Disk Acccess must support both sequential and direct/random access,
	\item We must provide Data Structures to mantain file location information on disk,
\end{itemize}

We identify the following Data Structures that are necessary in order to mantain the 
necessary information for our File System:

\paragraph{Boot Control Block}
A Boot Control Block is a data structure that is present on each volume, it 
contains information about how to boot up the system.

It is also known as the \emph{Boot Block} 
or \emph{Partition Boot Sector} depending on the Operating System.

\paragraph{Volume Control Block}
A Volume Control Block is also a data structure that must be present on each volume,
it contains information such as the partition table, number of blocks on each filesystem,
pointers to free blocks and free FCB blocks.

It is also known as the \emph{Master File Table} in UNIX or the \emph{Superblock} in Windows.

\paragraph{Directory Structure}
A Directory Structure is present on a per-filesystem basis, it contains file names and 
pointers to FCBs.

UNIX uses inode numbers, NTFS uses a master file table.

\paragraph{File Control Block (FCB)}
A File Control Block is present on a per-file basis, it contains the metadata specific 
to that file.

UNIX stores this information in \emph{inodes}, NTFS does it in a master file table as a 
relational database structure

\subsection{File Control Blocks}
File Control Blocks (or File Descriptors) are not only used to specify file metadata, but also to provide 
information about the file's location on disk, it must be stored on disk along regular files.

A copy of each FCP/FD is also stored in the Operating System's Global Open File Table.

\section{In-Memory Data Structures}
There is a set of data structures that are present in main memory which are 
necessary in order for the functioning of the Operating System's Filesystem, 
these are:

\begin{multicols}{2}
	\begin{itemize}
		\item The Mount Table.
		\item The Global Open File Table.
		\item The Directory Cache.
		\item The Local Open File Table.
	\end{itemize}
\end{multicols}

\paragraph{Open Files Tables}
We have encountered both the Local and the Global Open File Tables previously, 
they are both responsible for holding infomration regarding file that are being 
currently manipulated: the \emph{Global} table contains a copy of the FCB 
for every currently open file in the system while the \emph{Local} table contains a
pointer to the global table's entry for that specific file.

\section{Filesystem Implementations}

\subsection{Directories}
Directories are a fundamental structure of Operating Systems, as such, they 
must provide fast implementations for search, insert and delete operations while 
wasting a minimal amount of disk space, common data structures that come to our mind are:

\begin{itemize}
	\item A Linear List: Most of the methods can be impemented in O(n)\footnote{Which in this case 
	is pretty inefficient.}, 
	\item Hashtable: Usually implemented in addition to a linear structure to speed up searches. 
\end{itemize}

\subsection{File Allocation Methods}
Due to the 90/10 rule, the majority of files in a system are typically very small, while 
the vast majority of a disk's space is taken up by few but very large files.

Our I/O operation target both small and large files and we wish to keep 
the per-file cost low (by handling large files efficiently)

For this we have several Options:
\subsubsection{Option I, Contiguous Allocation}
In a similar fashion to the first approach that we provided for main memory allocation, 
we can trivially propose a solution which makes use of disk memory.

In this solution the OS keeps a list of free disk blocks, it allocates an appropriate 
sequence whne a file is created, the descriptor of that file must only store the start
location and it's size.

\begin{itemize}
	\item PROs: The implementation is very simple and is the best possible choice for 
	sequential access.
	\item CONs: Hard to change file size, fragmentation \emph{is} an issue.
\end{itemize}

\subsubsection{Option II, Linked Files}
The Operating System keeps a Linked List of free (non-contiguous) blocks, it also 
keeps a Linked List of where subsequent blocks are located. this frees the file 
from the requirement of physical sequentiality.

The descriptor must only keep a pointer to the head of the list, while each node in the 
list must only keep a pointer to the next block in each sector

\begin{itemize}
	\item PROs: No fragmentation, File changes are very easy.
	\item CONs: Both Sequential and Random accesses are very inefficient.
\end{itemize}

\subsubsection{Option III, Indexed Files}
\marginnote{This solution is sparsely used, such as in Nachos, an academic \OS with a Monolithic Kernel}
The File Descriptor contains a block of pointers, the user of the OS must 
declare the file's maximum length on it's creation.

The \OS allocates an array to hold the pointers to all the blocks 
when it creates a file, allocating the block on-demand as it's size increases.

\begin{itemize}
	\item PROs: No Fragmentation, Efficient Random Acces.
	\item CONs: File Descriptor becomes cumbersome, Flexibility is really poor, Unefficient Sequential Access.
\end{itemize}


\subsubsection{Option IV, Multi-Level Indexed Files}
Each file descriptor contains a number of block pointers of an arbitrary size $n$, the 
first $n-2$ hold pointers to data blocks, while pointers $n-1$ and $n$ point to another block 
of a much bigger size\footnote{E.g: the first one holds 12 pointers, the second `layer' has 10 blocks with 1024 pointers each.}

By Increasing exponentially after the first layer, a Multi-Level Indexed File increases it's 
size tremendously with very little overhead, given an FD with depth $layer$ where each layer as $n_p$ pointers:

\[
	Size = {(n_{p})}^{layer}	
\]

\begin{itemize}
	\item PROs: Simple to implement, Flexible, Optimized for small files.
	\item CONs: Not the most efficient (but not terrible), Lots of seek time due to non contiguous allocation.
\end{itemize}

\subsection{Free Space Management: Bitmaps}
In order to implement any system of File Indexing, we must often posses a data structure that 
holds information relative to the currently available free space in our system, we also 
need this data structure to be efficient enough as to allow us to locate and release space quickly.

The Bitmap has one bit for each block on the disk, 
if the bit is one the block is free, otherise the block is allocated.

Using a 32-bit bitmap one can quickly determine if any block in the next 32 is free by 
comparing it to 0, one can then use a bitwise instruction to find an empty block.

\paragraph{Bitmap Size}
Bitmaps are expensive: a $\SI{2}{\tera\byte}$ disk with 512-byte sector needs 
approximatively $\SI{500}{\mega\byte}$ of space.

\paragraph{Free Space Management: Linked List}
If most of the disk is in use, it will be expensive to find free blocks with a bitmap, an 
alternative implementation is to link the free blocks together, caching the head of the list 
in kernel memory.

With this implementation, Each block will then contain a pointer to the next free block, Allocations 
and De-Allocations are performed by modifying pointers of this list.

\section{Final Summary}
In this Lecture, We have seen that:

\begin{itemize}
	\item Many of the concerns of File System Implementation are similar to those 
	of Virtual Memory Implementation:
	\begin{itemize}
		\item Contiguous Allocation is simple but it suffers from fragmentation and 
		poor flexibility.
		\item Indexed Allocation is very similar to page tables.
	\end{itemize}
	\item Free Space can be tracked through either a bitmap or a linked list.
\end{itemize}
\end{document}
